% Generated by Sphinx.
\def\sphinxdocclass{report}
\documentclass[letterpaper,10pt,english]{sphinxmanual}
\usepackage[utf8]{inputenc}
\DeclareUnicodeCharacter{00A0}{\nobreakspace}
\usepackage{cmap}
\usepackage[T1]{fontenc}
\usepackage{babel}
\usepackage{times}
\usepackage[Bjarne]{fncychap}
\usepackage{longtable}
\usepackage{sphinx}
\usepackage{multirow}


\title{LBD Backend Documentation}
\date{January 14, 2015}
\release{Dev}
\author{LBD Development}
\newcommand{\sphinxlogo}{}
\renewcommand{\releasename}{Release}
\makeindex

\makeatletter
\def\PYG@reset{\let\PYG@it=\relax \let\PYG@bf=\relax%
    \let\PYG@ul=\relax \let\PYG@tc=\relax%
    \let\PYG@bc=\relax \let\PYG@ff=\relax}
\def\PYG@tok#1{\csname PYG@tok@#1\endcsname}
\def\PYG@toks#1+{\ifx\relax#1\empty\else%
    \PYG@tok{#1}\expandafter\PYG@toks\fi}
\def\PYG@do#1{\PYG@bc{\PYG@tc{\PYG@ul{%
    \PYG@it{\PYG@bf{\PYG@ff{#1}}}}}}}
\def\PYG#1#2{\PYG@reset\PYG@toks#1+\relax+\PYG@do{#2}}

\expandafter\def\csname PYG@tok@gd\endcsname{\def\PYG@tc##1{\textcolor[rgb]{0.63,0.00,0.00}{##1}}}
\expandafter\def\csname PYG@tok@gu\endcsname{\let\PYG@bf=\textbf\def\PYG@tc##1{\textcolor[rgb]{0.50,0.00,0.50}{##1}}}
\expandafter\def\csname PYG@tok@gt\endcsname{\def\PYG@tc##1{\textcolor[rgb]{0.00,0.27,0.87}{##1}}}
\expandafter\def\csname PYG@tok@gs\endcsname{\let\PYG@bf=\textbf}
\expandafter\def\csname PYG@tok@gr\endcsname{\def\PYG@tc##1{\textcolor[rgb]{1.00,0.00,0.00}{##1}}}
\expandafter\def\csname PYG@tok@cm\endcsname{\let\PYG@it=\textit\def\PYG@tc##1{\textcolor[rgb]{0.25,0.50,0.56}{##1}}}
\expandafter\def\csname PYG@tok@vg\endcsname{\def\PYG@tc##1{\textcolor[rgb]{0.73,0.38,0.84}{##1}}}
\expandafter\def\csname PYG@tok@m\endcsname{\def\PYG@tc##1{\textcolor[rgb]{0.13,0.50,0.31}{##1}}}
\expandafter\def\csname PYG@tok@mh\endcsname{\def\PYG@tc##1{\textcolor[rgb]{0.13,0.50,0.31}{##1}}}
\expandafter\def\csname PYG@tok@cs\endcsname{\def\PYG@tc##1{\textcolor[rgb]{0.25,0.50,0.56}{##1}}\def\PYG@bc##1{\setlength{\fboxsep}{0pt}\colorbox[rgb]{1.00,0.94,0.94}{\strut ##1}}}
\expandafter\def\csname PYG@tok@ge\endcsname{\let\PYG@it=\textit}
\expandafter\def\csname PYG@tok@vc\endcsname{\def\PYG@tc##1{\textcolor[rgb]{0.73,0.38,0.84}{##1}}}
\expandafter\def\csname PYG@tok@il\endcsname{\def\PYG@tc##1{\textcolor[rgb]{0.13,0.50,0.31}{##1}}}
\expandafter\def\csname PYG@tok@go\endcsname{\def\PYG@tc##1{\textcolor[rgb]{0.20,0.20,0.20}{##1}}}
\expandafter\def\csname PYG@tok@cp\endcsname{\def\PYG@tc##1{\textcolor[rgb]{0.00,0.44,0.13}{##1}}}
\expandafter\def\csname PYG@tok@gi\endcsname{\def\PYG@tc##1{\textcolor[rgb]{0.00,0.63,0.00}{##1}}}
\expandafter\def\csname PYG@tok@gh\endcsname{\let\PYG@bf=\textbf\def\PYG@tc##1{\textcolor[rgb]{0.00,0.00,0.50}{##1}}}
\expandafter\def\csname PYG@tok@ni\endcsname{\let\PYG@bf=\textbf\def\PYG@tc##1{\textcolor[rgb]{0.84,0.33,0.22}{##1}}}
\expandafter\def\csname PYG@tok@nl\endcsname{\let\PYG@bf=\textbf\def\PYG@tc##1{\textcolor[rgb]{0.00,0.13,0.44}{##1}}}
\expandafter\def\csname PYG@tok@nn\endcsname{\let\PYG@bf=\textbf\def\PYG@tc##1{\textcolor[rgb]{0.05,0.52,0.71}{##1}}}
\expandafter\def\csname PYG@tok@no\endcsname{\def\PYG@tc##1{\textcolor[rgb]{0.38,0.68,0.84}{##1}}}
\expandafter\def\csname PYG@tok@na\endcsname{\def\PYG@tc##1{\textcolor[rgb]{0.25,0.44,0.63}{##1}}}
\expandafter\def\csname PYG@tok@nb\endcsname{\def\PYG@tc##1{\textcolor[rgb]{0.00,0.44,0.13}{##1}}}
\expandafter\def\csname PYG@tok@nc\endcsname{\let\PYG@bf=\textbf\def\PYG@tc##1{\textcolor[rgb]{0.05,0.52,0.71}{##1}}}
\expandafter\def\csname PYG@tok@nd\endcsname{\let\PYG@bf=\textbf\def\PYG@tc##1{\textcolor[rgb]{0.33,0.33,0.33}{##1}}}
\expandafter\def\csname PYG@tok@ne\endcsname{\def\PYG@tc##1{\textcolor[rgb]{0.00,0.44,0.13}{##1}}}
\expandafter\def\csname PYG@tok@nf\endcsname{\def\PYG@tc##1{\textcolor[rgb]{0.02,0.16,0.49}{##1}}}
\expandafter\def\csname PYG@tok@si\endcsname{\let\PYG@it=\textit\def\PYG@tc##1{\textcolor[rgb]{0.44,0.63,0.82}{##1}}}
\expandafter\def\csname PYG@tok@s2\endcsname{\def\PYG@tc##1{\textcolor[rgb]{0.25,0.44,0.63}{##1}}}
\expandafter\def\csname PYG@tok@vi\endcsname{\def\PYG@tc##1{\textcolor[rgb]{0.73,0.38,0.84}{##1}}}
\expandafter\def\csname PYG@tok@nt\endcsname{\let\PYG@bf=\textbf\def\PYG@tc##1{\textcolor[rgb]{0.02,0.16,0.45}{##1}}}
\expandafter\def\csname PYG@tok@nv\endcsname{\def\PYG@tc##1{\textcolor[rgb]{0.73,0.38,0.84}{##1}}}
\expandafter\def\csname PYG@tok@s1\endcsname{\def\PYG@tc##1{\textcolor[rgb]{0.25,0.44,0.63}{##1}}}
\expandafter\def\csname PYG@tok@gp\endcsname{\let\PYG@bf=\textbf\def\PYG@tc##1{\textcolor[rgb]{0.78,0.36,0.04}{##1}}}
\expandafter\def\csname PYG@tok@sh\endcsname{\def\PYG@tc##1{\textcolor[rgb]{0.25,0.44,0.63}{##1}}}
\expandafter\def\csname PYG@tok@ow\endcsname{\let\PYG@bf=\textbf\def\PYG@tc##1{\textcolor[rgb]{0.00,0.44,0.13}{##1}}}
\expandafter\def\csname PYG@tok@sx\endcsname{\def\PYG@tc##1{\textcolor[rgb]{0.78,0.36,0.04}{##1}}}
\expandafter\def\csname PYG@tok@bp\endcsname{\def\PYG@tc##1{\textcolor[rgb]{0.00,0.44,0.13}{##1}}}
\expandafter\def\csname PYG@tok@c1\endcsname{\let\PYG@it=\textit\def\PYG@tc##1{\textcolor[rgb]{0.25,0.50,0.56}{##1}}}
\expandafter\def\csname PYG@tok@kc\endcsname{\let\PYG@bf=\textbf\def\PYG@tc##1{\textcolor[rgb]{0.00,0.44,0.13}{##1}}}
\expandafter\def\csname PYG@tok@c\endcsname{\let\PYG@it=\textit\def\PYG@tc##1{\textcolor[rgb]{0.25,0.50,0.56}{##1}}}
\expandafter\def\csname PYG@tok@mf\endcsname{\def\PYG@tc##1{\textcolor[rgb]{0.13,0.50,0.31}{##1}}}
\expandafter\def\csname PYG@tok@err\endcsname{\def\PYG@bc##1{\setlength{\fboxsep}{0pt}\fcolorbox[rgb]{1.00,0.00,0.00}{1,1,1}{\strut ##1}}}
\expandafter\def\csname PYG@tok@kd\endcsname{\let\PYG@bf=\textbf\def\PYG@tc##1{\textcolor[rgb]{0.00,0.44,0.13}{##1}}}
\expandafter\def\csname PYG@tok@ss\endcsname{\def\PYG@tc##1{\textcolor[rgb]{0.32,0.47,0.09}{##1}}}
\expandafter\def\csname PYG@tok@sr\endcsname{\def\PYG@tc##1{\textcolor[rgb]{0.14,0.33,0.53}{##1}}}
\expandafter\def\csname PYG@tok@mo\endcsname{\def\PYG@tc##1{\textcolor[rgb]{0.13,0.50,0.31}{##1}}}
\expandafter\def\csname PYG@tok@mi\endcsname{\def\PYG@tc##1{\textcolor[rgb]{0.13,0.50,0.31}{##1}}}
\expandafter\def\csname PYG@tok@kn\endcsname{\let\PYG@bf=\textbf\def\PYG@tc##1{\textcolor[rgb]{0.00,0.44,0.13}{##1}}}
\expandafter\def\csname PYG@tok@o\endcsname{\def\PYG@tc##1{\textcolor[rgb]{0.40,0.40,0.40}{##1}}}
\expandafter\def\csname PYG@tok@kr\endcsname{\let\PYG@bf=\textbf\def\PYG@tc##1{\textcolor[rgb]{0.00,0.44,0.13}{##1}}}
\expandafter\def\csname PYG@tok@s\endcsname{\def\PYG@tc##1{\textcolor[rgb]{0.25,0.44,0.63}{##1}}}
\expandafter\def\csname PYG@tok@kp\endcsname{\def\PYG@tc##1{\textcolor[rgb]{0.00,0.44,0.13}{##1}}}
\expandafter\def\csname PYG@tok@w\endcsname{\def\PYG@tc##1{\textcolor[rgb]{0.73,0.73,0.73}{##1}}}
\expandafter\def\csname PYG@tok@kt\endcsname{\def\PYG@tc##1{\textcolor[rgb]{0.56,0.13,0.00}{##1}}}
\expandafter\def\csname PYG@tok@sc\endcsname{\def\PYG@tc##1{\textcolor[rgb]{0.25,0.44,0.63}{##1}}}
\expandafter\def\csname PYG@tok@sb\endcsname{\def\PYG@tc##1{\textcolor[rgb]{0.25,0.44,0.63}{##1}}}
\expandafter\def\csname PYG@tok@k\endcsname{\let\PYG@bf=\textbf\def\PYG@tc##1{\textcolor[rgb]{0.00,0.44,0.13}{##1}}}
\expandafter\def\csname PYG@tok@se\endcsname{\let\PYG@bf=\textbf\def\PYG@tc##1{\textcolor[rgb]{0.25,0.44,0.63}{##1}}}
\expandafter\def\csname PYG@tok@sd\endcsname{\let\PYG@it=\textit\def\PYG@tc##1{\textcolor[rgb]{0.25,0.44,0.63}{##1}}}

\def\PYGZbs{\char`\\}
\def\PYGZus{\char`\_}
\def\PYGZob{\char`\{}
\def\PYGZcb{\char`\}}
\def\PYGZca{\char`\^}
\def\PYGZam{\char`\&}
\def\PYGZlt{\char`\<}
\def\PYGZgt{\char`\>}
\def\PYGZsh{\char`\#}
\def\PYGZpc{\char`\%}
\def\PYGZdl{\char`\$}
\def\PYGZhy{\char`\-}
\def\PYGZsq{\char`\'}
\def\PYGZdq{\char`\"}
\def\PYGZti{\char`\~}
% for compatibility with earlier versions
\def\PYGZat{@}
\def\PYGZlb{[}
\def\PYGZrb{]}
\makeatother

\begin{document}

\maketitle
\tableofcontents
\phantomsection\label{index::doc}


\textbf{Table of Contents}


\chapter{Requirements}
\label{requirements:requirements}\label{requirements::doc}\label{requirements:welcome-to-lbd-backend-s-documentation}
The back-end is written using Python 2.7. While it is considered to be legacy and (if we are mean) old, it is quite
widely spread and common version of Python. The good thing is that it is stable and does not change anymore, while
Python 3 is still in active development and has not yet seen the final form. Maybe at some point, if this software
is developed further, we will move from 2.7 to 3.x :) .

As database we decided to try out MongoDB, a NoSQL database that stores information as JSON documents. The reason for
this was to try out something other than SQL and because MongoDB has support for geospatial data builtin, which made
the use of GeoJSON easier. In addition we could also form geospatial indexes and searches without any additional
plugins. To use the MongoDB database from Python code, mongoengine and pymongo libraries are used. These can be installed
with Pip.

Software requirements:
\begin{itemize}
\item {} 
\textbf{MongoDB 2.6.6} or greater (designed with 2.6.6)

\item {} 
\textbf{Python 2.7.x} (designed with 2.7.8)
\begin{itemize}
\item {} 
\textbf{pymongo 2.7.2} or greater (designed with 2.7.2) (\emph{Python library})

\item {} 
\textbf{mongoengine 0.8.7} or greater (designed with 0.8.7) (\emph{Python library})

\item {} 
\textbf{httplib2 0.9} or greater (designed with 0.9) (\emph{Python library})

\end{itemize}

\end{itemize}


\chapter{Installation}
\label{installation:installation}\label{installation::doc}
This chapter describes where one can download the software and libraries required by the back-end and how they can be
installed.


\section{MongoDB}
\label{installation:mongodb}
MongoDB can be downloaded from \href{http://www.mongodb.com/downloads}{http://www.mongodb.com/downloads} for different operating systems. For Linux distributions
it might be available from package management system (e.g. Portage on Gentoo) depending on the architecture running the operating
system.

MongoDB installation guides for different operating systems can be found from \href{http://docs.mongodb.org/manual/installation/}{http://docs.mongodb.org/manual/installation/} .


\section{Python}
\label{installation:python}
Python can be downloaded from \href{https://www.python.org/downloads/}{https://www.python.org/downloads/} for multiple operating systems. As with MongoDB, on Linux
it can also be installed with package manager.


\subsection{Pip}
\label{installation:pip}
PIP is recommended for easy installation of Python libraries. Now for those using the latest versions of Python, there
are some good news, as PIP comes with the installation and on Linux of course it might be installed by the package manager.
However, if you are not lucky enough to have \textbf{Python 2.7.9} or running Linux system, you need to do this yourself.

So... if you have Python \textless{}= 2.7.8 or otherwise do not have PIP yet, follow the instructions at \href{https://pip.pypa.io/en/latest/installing.html}{https://pip.pypa.io/en/latest/installing.html} .

After the installation, the latest versions of libraries can be installed by running:

\begin{Verbatim}[commandchars=\\\{\}]
pip install \PYGZlt{}package name\PYGZgt{}
\end{Verbatim}

or specific version:

\begin{Verbatim}[commandchars=\\\{\}]
pip install \PYGZlt{}package name\PYGZgt{}==x.y.z
\end{Verbatim}


\section{Django}
\label{installation:django}
The recommended way to install django in all operating systems is through PIP:

\begin{Verbatim}[commandchars=\\\{\}]
pip install django
\end{Verbatim}

Of course nothing prohibits one to install it with Linux package manager or otherwise. More on django installation at
\href{https://docs.djangoproject.com/en/1.7/topics/install/}{https://docs.djangoproject.com/en/1.7/topics/install/} .


\chapter{Headers and CORS}
\label{headersandcors:headers-and-cors}\label{headersandcors::doc}

\section{Headers}
\label{headersandcors:headers}
The back-end uses two un-standard headers for user authentication and authorization. Both of these are required for most
functionalities.

\begin{tabulary}{\linewidth}{|L|L|}
\hline
\textsf{\relax 
Header name
} & \textsf{\relax 
Explanation
}\\
\hline
LBD\_LOGIN\_HEADER
 & 
Google OAuth authentication token.
\\

LBD\_OAUTH\_ID
 & 
Google id.
\\
\hline\end{tabulary}



\section{CORS}
\label{headersandcors:cors}
CORS or Cross-origin resource sharing mechanism allows resources to be requested from other domains. In order to be able to
use the back-end for example with AngularJS from another domain, some headers needed to be added. This is done in cors-middleware.

Cors middleware adds support for HTTP OPTIONS method and adds \emph{Access-Control-Allow-Origin}, \emph{Access-Control-Allow-Credentials} and
\emph{Access-Control-Allow-Headers} to the request.


\chapter{JSON Formats}
\label{json/jsondoc::doc}\label{json/jsondoc:json-formats}

\section{Location data}
\label{json/locationdatajson:locationjson}\label{json/locationdatajson::doc}\label{json/locationdatajson:location-data}
Location data uses GeoJSON (\href{http://geojson.org/}{http://geojson.org/}) specification, so all objects retrieved from the back-end are compatible
with GeoJSON readers. Thanks to GeoJSON's flexibility, new elements can be added to the objects. Back-end utilizes this
by adding ``metadata'' element inside ``properties'' element. This new field is completely optional and may contain information
such as status of the location data object, who has modified the metadata and so on. In this chapter this new element is
described in detail.


\subsection{JSON Format}
\label{json/locationdatajson:json-format}
Like stated above, the JSON used by the software follows the GeoJSON specification.

An example from \href{http://gejson.org/geojson-spec.html}{http://gejson.org/geojson-spec.html} (referenced 11.1.2015):

\begin{Verbatim}[commandchars=\\\{\}]
\PYG{p}{\PYGZob{}}
    \PYG{l+s}{\PYGZdq{}}\PYG{l+s}{type}\PYG{l+s}{\PYGZdq{}}\PYG{p}{:} \PYG{l+s}{\PYGZdq{}}\PYG{l+s}{FeatureCollection}\PYG{l+s}{\PYGZdq{}}\PYG{p}{,}
    \PYG{l+s}{\PYGZdq{}}\PYG{l+s}{features}\PYG{l+s}{\PYGZdq{}}\PYG{p}{:} \PYG{p}{[}
        \PYG{p}{\PYGZob{}}
            \PYG{l+s}{\PYGZdq{}}\PYG{l+s}{type}\PYG{l+s}{\PYGZdq{}}\PYG{p}{:} \PYG{l+s}{\PYGZdq{}}\PYG{l+s}{Feature}\PYG{l+s}{\PYGZdq{}}\PYG{p}{,}
            \PYG{l+s}{\PYGZdq{}}\PYG{l+s}{geometry}\PYG{l+s}{\PYGZdq{}}\PYG{p}{:} \PYG{p}{\PYGZob{}}
                \PYG{l+s}{\PYGZdq{}}\PYG{l+s}{type}\PYG{l+s}{\PYGZdq{}}\PYG{p}{:} \PYG{l+s}{\PYGZdq{}}\PYG{l+s}{Point}\PYG{l+s}{\PYGZdq{}}\PYG{p}{,}
                \PYG{l+s}{\PYGZdq{}}\PYG{l+s}{coordinates}\PYG{l+s}{\PYGZdq{}}\PYG{p}{:} \PYG{p}{[}\PYG{l+m+mf}{102.0}\PYG{p}{,} \PYG{l+m+mf}{0.5}\PYG{p}{]}
            \PYG{p}{\PYGZcb{}}\PYG{p}{,}
            \PYG{l+s}{\PYGZdq{}}\PYG{l+s}{properties}\PYG{l+s}{\PYGZdq{}}\PYG{p}{:} \PYG{p}{\PYGZob{}}
                \PYG{l+s}{\PYGZdq{}}\PYG{l+s}{prop0}\PYG{l+s}{\PYGZdq{}}\PYG{p}{:} \PYG{l+s}{\PYGZdq{}}\PYG{l+s}{value0}\PYG{l+s}{\PYGZdq{}}
            \PYG{p}{\PYGZcb{}}
        \PYG{p}{\PYGZcb{}}\PYG{p}{,}
        \PYG{p}{\PYGZob{}}
            \PYG{l+s}{\PYGZdq{}}\PYG{l+s}{type}\PYG{l+s}{\PYGZdq{}}\PYG{p}{:} \PYG{l+s}{\PYGZdq{}}\PYG{l+s}{Feature}\PYG{l+s}{\PYGZdq{}}\PYG{p}{,}
            \PYG{l+s}{\PYGZdq{}}\PYG{l+s}{geometry}\PYG{l+s}{\PYGZdq{}}\PYG{p}{:} \PYG{p}{\PYGZob{}}
                \PYG{l+s}{\PYGZdq{}}\PYG{l+s}{type}\PYG{l+s}{\PYGZdq{}}\PYG{p}{:} \PYG{l+s}{\PYGZdq{}}\PYG{l+s}{LineString}\PYG{l+s}{\PYGZdq{}}\PYG{p}{,}
                \PYG{l+s}{\PYGZdq{}}\PYG{l+s}{coordinates}\PYG{l+s}{\PYGZdq{}}\PYG{p}{:} \PYG{p}{[}
                    \PYG{p}{[}\PYG{l+m+mf}{102.0}\PYG{p}{,} \PYG{l+m+mf}{0.0}\PYG{p}{]}\PYG{p}{,} \PYG{p}{[}\PYG{l+m+mf}{103.0}\PYG{p}{,} \PYG{l+m+mf}{1.0}\PYG{p}{]}\PYG{p}{,} \PYG{p}{[}\PYG{l+m+mf}{104.0}\PYG{p}{,} \PYG{l+m+mf}{0.0}\PYG{p}{]}\PYG{p}{,} \PYG{p}{[}\PYG{l+m+mf}{105.0}\PYG{p}{,} \PYG{l+m+mf}{1.0}\PYG{p}{]}
                \PYG{p}{]}
            \PYG{p}{\PYGZcb{}}\PYG{p}{,}
            \PYG{l+s}{\PYGZdq{}}\PYG{l+s}{properties}\PYG{l+s}{\PYGZdq{}}\PYG{p}{:} \PYG{p}{\PYGZob{}}
                \PYG{l+s}{\PYGZdq{}}\PYG{l+s}{prop0}\PYG{l+s}{\PYGZdq{}}\PYG{p}{:} \PYG{l+s}{\PYGZdq{}}\PYG{l+s}{value0}\PYG{l+s}{\PYGZdq{}}\PYG{p}{,}
                \PYG{l+s}{\PYGZdq{}}\PYG{l+s}{prop1}\PYG{l+s}{\PYGZdq{}}\PYG{p}{:} \PYG{l+m+mf}{0.0}
            \PYG{p}{\PYGZcb{}}
        \PYG{p}{\PYGZcb{}}\PYG{p}{,}
        \PYG{p}{\PYGZob{}}
            \PYG{l+s}{\PYGZdq{}}\PYG{l+s}{type}\PYG{l+s}{\PYGZdq{}}\PYG{p}{:} \PYG{l+s}{\PYGZdq{}}\PYG{l+s}{Feature}\PYG{l+s}{\PYGZdq{}}\PYG{p}{,}
            \PYG{l+s}{\PYGZdq{}}\PYG{l+s}{geometry}\PYG{l+s}{\PYGZdq{}}\PYG{p}{:} \PYG{p}{\PYGZob{}}
                \PYG{l+s}{\PYGZdq{}}\PYG{l+s}{type}\PYG{l+s}{\PYGZdq{}}\PYG{p}{:} \PYG{l+s}{\PYGZdq{}}\PYG{l+s}{Polygon}\PYG{l+s}{\PYGZdq{}}\PYG{p}{,}
                \PYG{l+s}{\PYGZdq{}}\PYG{l+s}{coordinates}\PYG{l+s}{\PYGZdq{}}\PYG{p}{:} \PYG{p}{[}
                     \PYG{p}{[} \PYG{p}{[}\PYG{l+m+mf}{100.0}\PYG{p}{,} \PYG{l+m+mf}{0.0}\PYG{p}{]}\PYG{p}{,} \PYG{p}{[}\PYG{l+m+mf}{101.0}\PYG{p}{,} \PYG{l+m+mf}{0.0}\PYG{p}{]}\PYG{p}{,} \PYG{p}{[}\PYG{l+m+mf}{101.0}\PYG{p}{,} \PYG{l+m+mf}{1.0}\PYG{p}{]}\PYG{p}{,}
                       \PYG{p}{[}\PYG{l+m+mf}{100.0}\PYG{p}{,} \PYG{l+m+mf}{1.0}\PYG{p}{]}\PYG{p}{,} \PYG{p}{[}\PYG{l+m+mf}{100.0}\PYG{p}{,} \PYG{l+m+mf}{0.0}\PYG{p}{]} \PYG{p}{]}
                 \PYG{p}{]}
            \PYG{p}{\PYGZcb{}}\PYG{p}{,}
            \PYG{l+s}{\PYGZdq{}}\PYG{l+s}{properties}\PYG{l+s}{\PYGZdq{}}\PYG{p}{:} \PYG{p}{\PYGZob{}}
                \PYG{l+s}{\PYGZdq{}}\PYG{l+s}{prop0}\PYG{l+s}{\PYGZdq{}}\PYG{p}{:} \PYG{l+s}{\PYGZdq{}}\PYG{l+s}{value0}\PYG{l+s}{\PYGZdq{}}\PYG{p}{,}
                \PYG{l+s}{\PYGZdq{}}\PYG{l+s}{prop1}\PYG{l+s}{\PYGZdq{}}\PYG{p}{:} \PYG{p}{\PYGZob{}}
                    \PYG{l+s}{\PYGZdq{}}\PYG{l+s}{this}\PYG{l+s}{\PYGZdq{}}\PYG{p}{:} \PYG{l+s}{\PYGZdq{}}\PYG{l+s}{that}\PYG{l+s}{\PYGZdq{}}
                \PYG{p}{\PYGZcb{}}
            \PYG{p}{\PYGZcb{}}
        \PYG{p}{\PYGZcb{}}
    \PYG{p}{]}
\PYG{p}{\PYGZcb{}}
\end{Verbatim}

\begin{notice}{note}{Note:}
It should be noted that the current system does not support other feature types than ``Point''.
\end{notice}

Back-end supports GeoJSON Feature when relaying information on single object and FeatureCollection when sending multiple
objects.

The format for the additional information is:

\begin{tabulary}{\linewidth}{|L|L|L|L|}
\hline
\textsf{\relax 
Field's name
} & \textsf{\relax 
Required
} & \textsf{\relax 
Value type
} & \textsf{\relax 
Notes
}\\
\hline
status
 & 
True
 & 
String
 & \\

modified
 & 
True
 & 
String
 & 
Unix timestamp (in seconds). Tells when the metadata was modified
\\

modifier
 & 
True
 & 
String
 & 
Tells who has modified the metadata
\\

info
 & 
True
 & 
String
 & \\
\hline\end{tabulary}


Which translates to:

\begin{Verbatim}[commandchars=\\\{\}]
\PYGZdq{}metadata\PYGZdq{}: \PYGZob{}
    \PYGZdq{}status\PYGZdq{}: **String**,
    \PYGZdq{}modified\PYGZdq{}: **Integer**,
    \PYGZdq{}modifier\PYGZdq{}: **String**,
    \PYGZdq{}info\PYGZdq{}: **String**
\PYGZcb{}
\end{Verbatim}

And a ``real'' example using open data service of Tampere (\href{http://www.tampere.fi/tampereinfo/avoindata.html}{http://www.tampere.fi/tampereinfo/avoindata.html}) to provide
the Streetlight information:

\begin{Verbatim}[commandchars=\\\{\}]
\PYG{p}{\PYGZob{}}
    \PYG{l+s}{\PYGZdq{}}\PYG{l+s}{geometry}\PYG{l+s}{\PYGZdq{}}\PYG{p}{:} \PYG{p}{\PYGZob{}}
        \PYG{l+s}{\PYGZdq{}}\PYG{l+s}{type}\PYG{l+s}{\PYGZdq{}}\PYG{p}{:} \PYG{l+s}{\PYGZdq{}}\PYG{l+s}{Point}\PYG{l+s}{\PYGZdq{}}\PYG{p}{,}
        \PYG{l+s}{\PYGZdq{}}\PYG{l+s}{coordinates}\PYG{l+s}{\PYGZdq{}}\PYG{p}{:} \PYG{p}{[}\PYG{l+m+mf}{23.643239226767022}\PYG{p}{,} \PYG{l+m+mf}{61.519112683582854}\PYG{p}{]}
    \PYG{p}{\PYGZcb{}}\PYG{p}{,}
    \PYG{l+s}{\PYGZdq{}}\PYG{l+s}{id}\PYG{l+s}{\PYGZdq{}}\PYG{p}{:} \PYG{l+s}{\PYGZdq{}}\PYG{l+s}{WFS\PYGZus{}KATUVALO.405172}\PYG{l+s}{\PYGZdq{}}\PYG{p}{,}
    \PYG{l+s}{\PYGZdq{}}\PYG{l+s}{type}\PYG{l+s}{\PYGZdq{}}\PYG{p}{:} \PYG{l+s}{\PYGZdq{}}\PYG{l+s}{Feature}\PYG{l+s}{\PYGZdq{}}\PYG{p}{,}
    \PYG{l+s}{\PYGZdq{}}\PYG{l+s}{properties}\PYG{l+s}{\PYGZdq{}}\PYG{p}{:} \PYG{p}{\PYGZob{}}
        \PYG{l+s}{\PYGZdq{}}\PYG{l+s}{NIMI}\PYG{l+s}{\PYGZdq{}}\PYG{p}{:} \PYG{l+s}{\PYGZdq{}}\PYG{l+s}{XPWR\PYGZus{}6769212}\PYG{l+s}{\PYGZdq{}}\PYG{p}{,}
        \PYG{l+s}{\PYGZdq{}}\PYG{l+s}{LAMPPU\PYGZus{}TYYPPI\PYGZus{}KOODI}\PYG{l+s}{\PYGZdq{}}\PYG{p}{:} \PYG{l+s}{\PYGZdq{}}\PYG{l+s}{100340}\PYG{l+s}{\PYGZdq{}}\PYG{p}{,}
        \PYG{l+s}{\PYGZdq{}}\PYG{l+s}{TYYPPI\PYGZus{}KOODI}\PYG{l+s}{\PYGZdq{}}\PYG{p}{:} \PYG{l+s}{\PYGZdq{}}\PYG{l+s}{105007}\PYG{l+s}{\PYGZdq{}}\PYG{p}{,}
        \PYG{l+s}{\PYGZdq{}}\PYG{l+s}{KATUVALO\PYGZus{}ID}\PYG{l+s}{\PYGZdq{}}\PYG{p}{:} \PYG{l+m+mi}{405172}\PYG{p}{,}
        \PYG{l+s}{\PYGZdq{}}\PYG{l+s}{LAMPPU\PYGZus{}TYYPPI}\PYG{l+s}{\PYGZdq{}}\PYG{p}{:} \PYG{l+s}{\PYGZdq{}}\PYG{l+s}{ST 100 (SIEMENS)}\PYG{l+s}{\PYGZdq{}}\PYG{p}{,}
        \PYG{l+s}{\PYGZdq{}}\PYG{l+s}{metadata}\PYG{l+s}{\PYGZdq{}}\PYG{p}{:} \PYG{p}{\PYGZob{}}
            \PYG{l+s}{\PYGZdq{}}\PYG{l+s}{status}\PYG{l+s}{\PYGZdq{}}\PYG{p}{:} \PYG{l+s}{\PYGZdq{}}\PYG{l+s}{foobar}\PYG{l+s}{\PYGZdq{}}\PYG{p}{,}
            \PYG{l+s}{\PYGZdq{}}\PYG{l+s}{note}\PYG{l+s}{\PYGZdq{}}\PYG{p}{:} \PYG{l+s}{\PYGZdq{}}\PYG{l+s}{FOOBAR}\PYG{l+s}{\PYGZdq{}}\PYG{p}{,}
            \PYG{l+s}{\PYGZdq{}}\PYG{l+s}{modifier}\PYG{l+s}{\PYGZdq{}}\PYG{p}{:} \PYG{l+s}{\PYGZdq{}}\PYG{l+s}{tiina@teekkari.fi}\PYG{l+s}{\PYGZdq{}}\PYG{p}{,}
            \PYG{l+s}{\PYGZdq{}}\PYG{l+s}{modified}\PYG{l+s}{\PYGZdq{}}\PYG{p}{:} \PYG{l+m+mi}{1420741774}
        \PYG{p}{\PYGZcb{}}
    \PYG{p}{\PYGZcb{}}\PYG{p}{,}
    \PYG{l+s}{\PYGZdq{}}\PYG{l+s}{geometry\PYGZus{}name}\PYG{l+s}{\PYGZdq{}}\PYG{p}{:} \PYG{l+s}{\PYGZdq{}}\PYG{l+s}{GEOLOC}\PYG{l+s}{\PYGZdq{}}
\PYG{p}{\PYGZcb{}}
\end{Verbatim}


\section{Message}
\label{json/messagejson:messagejson}\label{json/messagejson:message}\label{json/messagejson::doc}
The JSON for messages has been influenced by GeoJSON (a little bit). This is visible when comparing how single and multiple
messages are relayed. A single message has type ``Message'' while multiple messages are sent in a JSON ``envelope'' with ``MessageCollection''
as type.

An example containing both single message and a collection of messages:

\begin{Verbatim}[commandchars=\\\{\}]
\PYG{p}{\PYGZob{}}
    \PYG{l+s}{\PYGZdq{}}\PYG{l+s}{messages}\PYG{l+s}{\PYGZdq{}}\PYG{p}{:} \PYG{p}{[}
        \PYG{p}{\PYGZob{}}
            \PYG{l+s}{\PYGZdq{}}\PYG{l+s}{category}\PYG{l+s}{\PYGZdq{}}\PYG{p}{:} \PYG{l+s}{\PYGZdq{}}\PYG{l+s}{Streetlights}\PYG{l+s}{\PYGZdq{}}\PYG{p}{,}
            \PYG{l+s}{\PYGZdq{}}\PYG{l+s}{attachments}\PYG{l+s}{\PYGZdq{}}\PYG{p}{:} \PYG{p}{[}
                \PYG{p}{\PYGZob{}}
                    \PYG{l+s}{\PYGZdq{}}\PYG{l+s}{category}\PYG{l+s}{\PYGZdq{}}\PYG{p}{:} \PYG{l+s}{\PYGZdq{}}\PYG{l+s}{Streetlights}\PYG{l+s}{\PYGZdq{}}\PYG{p}{,}
                    \PYG{l+s}{\PYGZdq{}}\PYG{l+s}{id}\PYG{l+s}{\PYGZdq{}}\PYG{p}{:} \PYG{l+s}{\PYGZdq{}}\PYG{l+s}{WFS\PYGZus{}KATUVALO.405172}\PYG{l+s}{\PYGZdq{}}
                \PYG{p}{\PYGZcb{}}
            \PYG{p}{]}\PYG{p}{,}
            \PYG{l+s}{\PYGZdq{}}\PYG{l+s}{type}\PYG{l+s}{\PYGZdq{}}\PYG{p}{:} \PYG{l+s}{\PYGZdq{}}\PYG{l+s}{Message}\PYG{l+s}{\PYGZdq{}}\PYG{p}{,}
            \PYG{l+s}{\PYGZdq{}}\PYG{l+s}{topic}\PYG{l+s}{\PYGZdq{}}\PYG{p}{:} \PYG{l+s}{\PYGZdq{}}\PYG{l+s}{Testi}\PYG{l+s}{\PYGZdq{}}\PYG{p}{,}
            \PYG{l+s}{\PYGZdq{}}\PYG{l+s}{messageread}\PYG{l+s}{\PYGZdq{}}\PYG{p}{:} \PYG{n}{true}\PYG{p}{,}
            \PYG{l+s}{\PYGZdq{}}\PYG{l+s}{message}\PYG{l+s}{\PYGZdq{}}\PYG{p}{:} \PYG{l+s}{\PYGZdq{}}\PYG{l+s}{Tämä on testiviesti}\PYG{l+s}{\PYGZdq{}}\PYG{p}{,}
            \PYG{l+s}{\PYGZdq{}}\PYG{l+s}{recipient}\PYG{l+s}{\PYGZdq{}}\PYG{p}{:} \PYG{l+s}{\PYGZdq{}}\PYG{l+s}{tiina@teekkari.com}\PYG{l+s}{\PYGZdq{}}\PYG{p}{,}
            \PYG{l+s}{\PYGZdq{}}\PYG{l+s}{id}\PYG{l+s}{\PYGZdq{}}\PYG{p}{:} \PYG{l+m+mi}{10}\PYG{p}{,}
            \PYG{l+s}{\PYGZdq{}}\PYG{l+s}{sender}\PYG{l+s}{\PYGZdq{}}\PYG{p}{:} \PYG{l+s}{\PYGZdq{}}\PYG{l+s}{tiina@teekkari.com}\PYG{l+s}{\PYGZdq{}}
        \PYG{p}{\PYGZcb{}}\PYG{p}{,}
        \PYG{p}{\PYGZob{}}
            \PYG{l+s}{\PYGZdq{}}\PYG{l+s}{category}\PYG{l+s}{\PYGZdq{}}\PYG{p}{:} \PYG{l+s}{\PYGZdq{}}\PYG{l+s}{Streetlights}\PYG{l+s}{\PYGZdq{}}\PYG{p}{,}
            \PYG{l+s}{\PYGZdq{}}\PYG{l+s}{attachments}\PYG{l+s}{\PYGZdq{}}\PYG{p}{:} \PYG{p}{[}
                \PYG{p}{\PYGZob{}}
                    \PYG{l+s}{\PYGZdq{}}\PYG{l+s}{category}\PYG{l+s}{\PYGZdq{}}\PYG{p}{:} \PYG{l+s}{\PYGZdq{}}\PYG{l+s}{Streetlights}\PYG{l+s}{\PYGZdq{}}\PYG{p}{,}
                    \PYG{l+s}{\PYGZdq{}}\PYG{l+s}{id}\PYG{l+s}{\PYGZdq{}}\PYG{p}{:} \PYG{l+s}{\PYGZdq{}}\PYG{l+s}{WFS\PYGZus{}KATUVALO.405172}\PYG{l+s}{\PYGZdq{}}
                \PYG{p}{\PYGZcb{}}
            \PYG{p}{]}\PYG{p}{,}
            \PYG{l+s}{\PYGZdq{}}\PYG{l+s}{timestamp}\PYG{l+s}{\PYGZdq{}}\PYG{p}{:} \PYG{l+m+mi}{1420786021}\PYG{p}{,}
            \PYG{l+s}{\PYGZdq{}}\PYG{l+s}{topic}\PYG{l+s}{\PYGZdq{}}\PYG{p}{:} \PYG{l+s}{\PYGZdq{}}\PYG{l+s}{Testi}\PYG{l+s}{\PYGZdq{}}\PYG{p}{,}
            \PYG{l+s}{\PYGZdq{}}\PYG{l+s}{messageread}\PYG{l+s}{\PYGZdq{}}\PYG{p}{:} \PYG{n}{true}\PYG{p}{,}
            \PYG{l+s}{\PYGZdq{}}\PYG{l+s}{type}\PYG{l+s}{\PYGZdq{}}\PYG{p}{:} \PYG{l+s}{\PYGZdq{}}\PYG{l+s}{Message}\PYG{l+s}{\PYGZdq{}}\PYG{p}{,}
            \PYG{l+s}{\PYGZdq{}}\PYG{l+s}{message}\PYG{l+s}{\PYGZdq{}}\PYG{p}{:} \PYG{l+s}{\PYGZdq{}}\PYG{l+s}{Tämä on testiviesti}\PYG{l+s}{\PYGZdq{}}\PYG{p}{,}
            \PYG{l+s}{\PYGZdq{}}\PYG{l+s}{recipient}\PYG{l+s}{\PYGZdq{}}\PYG{p}{:} \PYG{l+s}{\PYGZdq{}}\PYG{l+s}{tiina@teekkari.com}\PYG{l+s}{\PYGZdq{}}\PYG{p}{,}
            \PYG{l+s}{\PYGZdq{}}\PYG{l+s}{id}\PYG{l+s}{\PYGZdq{}}\PYG{p}{:} \PYG{l+m+mi}{11}\PYG{p}{,}
            \PYG{l+s}{\PYGZdq{}}\PYG{l+s}{sender}\PYG{l+s}{\PYGZdq{}}\PYG{p}{:} \PYG{l+s}{\PYGZdq{}}\PYG{l+s}{tiina@teekkari.com}\PYG{l+s}{\PYGZdq{}}
        \PYG{p}{\PYGZcb{}}
    \PYG{p}{]}\PYG{p}{,}
    \PYG{l+s}{\PYGZdq{}}\PYG{l+s}{type}\PYG{l+s}{\PYGZdq{}}\PYG{p}{:} \PYG{l+s}{\PYGZdq{}}\PYG{l+s}{MessageCollection}\PYG{l+s}{\PYGZdq{}}\PYG{p}{,}
    \PYG{l+s}{\PYGZdq{}}\PYG{l+s}{totalMessages}\PYG{l+s}{\PYGZdq{}}\PYG{p}{:} \PYG{l+m+mi}{2}
\PYG{p}{\PYGZcb{}}
\end{Verbatim}

Message collection has three elements:

\begin{tabulary}{\linewidth}{|L|L|L|L|}
\hline
\textsf{\relax 
Field's name
} & \textsf{\relax 
Required
} & \textsf{\relax 
Value type
} & \textsf{\relax 
Notes
}\\
\hline
type
 & 
True
 & 
String
 & 
Always ``MessageCollection''.
\\

messages
 & 
True
 & 
List
 & 
List of messages. Format below.
\\

totalMessages
 & 
True
 & 
Integer
 & 
Amount of messages in the collection.
\\
\hline\end{tabulary}


Format for a single message is following:

\begin{tabulary}{\linewidth}{|L|L|L|L|}
\hline
\textsf{\relax 
Field's name
} & \textsf{\relax 
Required
} & \textsf{\relax 
Value type
} & \textsf{\relax 
Notes
}\\
\hline
type
 & 
True
 & 
String
 & 
Always ``Message''
\\

topic
 & 
True
 & 
String
 & 
Topic of the message
\\

category
 & 
False
 & 
String
 & 
Category of the message. Name of a location data collection.
\\

id
 & 
True
 & 
Integer
 & 
Message id.
\\

sender
 & 
True
 & 
String
 & 
Tells who sent the message.
\\

recipient
 & 
True
 & 
String
 & 
Tells who is the recipient.
\\

message
 & 
True
 & 
String
 & 
Content of the message.
\\

timestamp
 & 
True
 & 
Integer
 & 
Unix timestamp (in seconds). Tells when the message was sent.
\\

attachments
 & 
False
 & 
List
 & 
List of attachments. Format described below.
\\
\hline\end{tabulary}


\begin{notice}{note}{Note:}
If the a category is specified for a message, there must exist a location data collection with that name.
\end{notice}

If attachment element is added to a message, message category becomes a required field. For flexibility, the message category
and the attachment category can be different from each other.

Attachment elements are JSON documents with two fields:

\begin{tabulary}{\linewidth}{|L|L|L|L|}
\hline
\textsf{\relax 
Fieldname
} & \textsf{\relax 
Required
} & \textsf{\relax 
Value type
} & \textsf{\relax 
Notes
}\\
\hline
category
 & 
True
 & 
String
 & 
Category of the attachment. Name of a location data collection.
\\

id
 & 
True
 & 
String
 & 
Id of the attached object. Must exist in the specified location data collection.
\\
\hline\end{tabulary}



\section{Search}
\label{json/search:searchjson}\label{json/search:search}\label{json/search::doc}
Search from the back-end is done by posting a JSON document to the back-end (see:
{\hyperref[restdoc:locationrest]{\emph{REST documentation}}}).

\begin{tabulary}{\linewidth}{|L|L|L|L|}
\hline
\textsf{\relax 
Field's name
} & \textsf{\relax 
Required
} & \textsf{\relax 
Value type
} & \textsf{\relax 
Notes
}\\
\hline
from
 & 
True
 & 
String
 & 
Only allowed value currently is ``ALL''
\\

search
 & 
True
 & 
String
 & 
Search phrase
\\

limit
 & 
True
 & 
Integer
 & 
Maximum number of results returned.
\\
\hline\end{tabulary}


Result format:

\begin{tabulary}{\linewidth}{|L|L|L|L|}
\hline
\textsf{\relax 
Field's name
} & \textsf{\relax 
Required
} & \textsf{\relax 
Value type
} & \textsf{\relax 
Notes
}\\
\hline
totalresults
 & 
True
 & 
Integer
 & 
The total amount of results.
\\

limit
 & 
True
 & 
Integer
 & 
The defined limit of the results.
\\

results
 & 
True
 & 
JSON
 & 
GeoJSON FeatureCollection containing the results.
\\
\hline\end{tabulary}


Example:

\begin{Verbatim}[commandchars=\\\{\}]
SEARCH:
\PYGZob{}
    \PYGZdq{}from\PYGZdq{}: \PYGZdq{}ALL\PYGZdq{},
    \PYGZdq{}search\PYGZdq{}: \PYGZdq{}42\PYGZdq{},
    \PYGZdq{}limit\PYGZdq{}: 21
\PYGZcb{}

RESULT:
\PYGZob{}
    \PYGZdq{}totalresults\PYGZdq{}: 1138,
    \PYGZdq{}limit\PYGZdq{}: 42,
    \PYGZdq{}results\PYGZdq{}: \PYGZob{} ...GeoJSON FeatureCollection... \PYGZcb{}
\PYGZcb{}
\end{Verbatim}


\chapter{REST documentation}
\label{restdoc::doc}\label{restdoc:rest-documentation}\label{restdoc:restdoc}
This chapter describes the REST APIs.

List of status codes used by the REST:

\begin{tabulary}{\linewidth}{|L|L|L|}
\hline
\textsf{\relax 
Status
} & \textsf{\relax 
Meaning
} & \textsf{\relax 
Notes
}\\
\hline
200
 & 
OK
 & 
Request successful
\\

401
 & 
Unauthorized
 & 
``Login'' failed.
\\

403
 & 
Forbidden
 & 
You shall not pass!
\\

404
 & 
Not Found
 & 
Resource not found.
\\

405
 & 
Method not allowed
 & 
HTTP method not allowed.
\\

418
 & 
I'm a teapot
 & 
Short and stout!
\\

500
 & 
Internal Server Error
 & 
Server snafu'd
\\
\hline\end{tabulary}



\section{Location data REST}
\label{restdoc:location-data-rest}\label{restdoc:locationrest}
Location data can be accessed from \textbf{/locationdata/api/} (e.g. \emph{www.example.com/locationdata/api/}). This URL does not yet require
authentication and return the installed location data services and what data (element names) they contain.

\begin{notice}{note}{Note:}
When creating or updating resources, only metadata is updated or created currently. It is not possible to create actual
location data objects... yet.
\end{notice}

In urls \textbf{\textless{}collection name\textgreater{}} and \textbf{\textless{}resource\textgreater{}} are to be replaced with appropriate values. Both are strings.


\subsection{/locationdata/api/}
\label{restdoc:locationdata-api}
Returns the installed location data services that can be accessed by appending the name of the service to the base url of
the location data api.

\textbf{Allowed methods:}
\begin{itemize}
\item {} 
GET

\end{itemize}

\textbf{URL parameters}
\begin{itemize}
\item {} 
None

\end{itemize}

Example result:

\begin{Verbatim}[commandchars=\\\{\}]
\PYG{p}{[}
    \PYG{p}{\PYGZob{}}
        \PYG{l+s}{\PYGZdq{}}\PYG{l+s}{fields}\PYG{l+s}{\PYGZdq{}}\PYG{p}{:} \PYG{p}{[}
            \PYG{l+s}{\PYGZdq{}}\PYG{l+s}{geometry\PYGZus{}name}\PYG{l+s}{\PYGZdq{}}\PYG{p}{,}
            \PYG{l+s}{\PYGZdq{}}\PYG{l+s}{geometry.type}\PYG{l+s}{\PYGZdq{}}\PYG{p}{,}
            \PYG{l+s}{\PYGZdq{}}\PYG{l+s}{geometry.coordinates}\PYG{l+s}{\PYGZdq{}}\PYG{p}{,}
            \PYG{l+s}{\PYGZdq{}}\PYG{l+s}{id}\PYG{l+s}{\PYGZdq{}}\PYG{p}{,}
            \PYG{l+s}{\PYGZdq{}}\PYG{l+s}{type}\PYG{l+s}{\PYGZdq{}}\PYG{p}{,}
            \PYG{l+s}{\PYGZdq{}}\PYG{l+s}{properties.URAKKA\PYGZus{}ALUE}\PYG{l+s}{\PYGZdq{}}\PYG{p}{,}
            \PYG{l+s}{\PYGZdq{}}\PYG{l+s}{properties.OSA\PYGZus{}ALUE\PYGZus{}NIMI}\PYG{l+s}{\PYGZdq{}}\PYG{p}{,}
            \PYG{l+s}{\PYGZdq{}}\PYG{l+s}{properties.PINTA\PYGZus{}ALA}\PYG{l+s}{\PYGZdq{}}\PYG{p}{,}
            \PYG{l+s}{\PYGZdq{}}\PYG{l+s}{properties.KAYTTOLK}\PYG{l+s}{\PYGZdq{}}\PYG{p}{,}
            \PYG{l+s}{\PYGZdq{}}\PYG{l+s}{properties.ALUE\PYGZus{}NIMI}\PYG{l+s}{\PYGZdq{}}\PYG{p}{,}
            \PYG{l+s}{\PYGZdq{}}\PYG{l+s}{properties.TILAAJA}\PYG{l+s}{\PYGZdq{}}\PYG{p}{,}
            \PYG{l+s}{\PYGZdq{}}\PYG{l+s}{properties.VIHERALUEEN\PYGZus{}OSAN\PYGZus{}ID}\PYG{l+s}{\PYGZdq{}}\PYG{p}{,}
            \PYG{l+s}{\PYGZdq{}}\PYG{l+s}{properties.KAUPUNGINOSA}\PYG{l+s}{\PYGZdq{}}\PYG{p}{,}
            \PYG{l+s}{\PYGZdq{}}\PYG{l+s}{properties.TOIMLK}\PYG{l+s}{\PYGZdq{}}\PYG{p}{,}
            \PYG{l+s}{\PYGZdq{}}\PYG{l+s}{properties.ALUE\PYGZus{}SIJ}\PYG{l+s}{\PYGZdq{}}\PYG{p}{,}
            \PYG{l+s}{\PYGZdq{}}\PYG{l+s}{properties.HOITOLK}\PYG{l+s}{\PYGZdq{}}
        \PYG{p}{]}\PYG{p}{,}
        \PYG{l+s}{\PYGZdq{}}\PYG{l+s}{name}\PYG{l+s}{\PYGZdq{}}\PYG{p}{:} \PYG{l+s}{\PYGZdq{}}\PYG{l+s}{Playgrounds}\PYG{l+s}{\PYGZdq{}}\PYG{p}{,}
        \PYG{l+s}{\PYGZdq{}}\PYG{l+s}{description}\PYG{l+s}{\PYGZdq{}}\PYG{p}{:} \PYG{l+s}{\PYGZdq{}}\PYG{l+s}{Ring around the rosie}\PYG{l+s}{\PYGZdq{}}
    \PYG{p}{\PYGZcb{}}\PYG{p}{,}
    \PYG{p}{\PYGZob{}}
        \PYG{l+s}{\PYGZdq{}}\PYG{l+s}{fields}\PYG{l+s}{\PYGZdq{}}\PYG{p}{:} \PYG{p}{[}
            \PYG{l+s}{\PYGZdq{}}\PYG{l+s}{geometry\PYGZus{}name}\PYG{l+s}{\PYGZdq{}}\PYG{p}{,}
            \PYG{l+s}{\PYGZdq{}}\PYG{l+s}{geometry.type}\PYG{l+s}{\PYGZdq{}}\PYG{p}{,}
            \PYG{l+s}{\PYGZdq{}}\PYG{l+s}{geometry.coordinates}\PYG{l+s}{\PYGZdq{}}\PYG{p}{,}
            \PYG{l+s}{\PYGZdq{}}\PYG{l+s}{id}\PYG{l+s}{\PYGZdq{}}\PYG{p}{,}
            \PYG{l+s}{\PYGZdq{}}\PYG{l+s}{type}\PYG{l+s}{\PYGZdq{}}\PYG{p}{,}
            \PYG{l+s}{\PYGZdq{}}\PYG{l+s}{properties.NIMI}\PYG{l+s}{\PYGZdq{}}\PYG{p}{,}
            \PYG{l+s}{\PYGZdq{}}\PYG{l+s}{properties.LAMPPU\PYGZus{}TYYPPI\PYGZus{}KOODI}\PYG{l+s}{\PYGZdq{}}\PYG{p}{,}
            \PYG{l+s}{\PYGZdq{}}\PYG{l+s}{properties.TYYPPI\PYGZus{}KOODI}\PYG{l+s}{\PYGZdq{}}\PYG{p}{,}
            \PYG{l+s}{\PYGZdq{}}\PYG{l+s}{properties.KATUVALO\PYGZus{}ID}\PYG{l+s}{\PYGZdq{}}\PYG{p}{,}
            \PYG{l+s}{\PYGZdq{}}\PYG{l+s}{properties.LAMPPU\PYGZus{}TYYPPI}\PYG{l+s}{\PYGZdq{}}\PYG{p}{,}
            \PYG{l+s}{\PYGZdq{}}\PYG{l+s}{properties.TYYPPI}\PYG{l+s}{\PYGZdq{}}
        \PYG{p}{]}\PYG{p}{,}
        \PYG{l+s}{\PYGZdq{}}\PYG{l+s}{name}\PYG{l+s}{\PYGZdq{}}\PYG{p}{:} \PYG{l+s}{\PYGZdq{}}\PYG{l+s}{Streetlights}\PYG{l+s}{\PYGZdq{}}\PYG{p}{,}
        \PYG{l+s}{\PYGZdq{}}\PYG{l+s}{description}\PYG{l+s}{\PYGZdq{}}\PYG{p}{:} \PYG{l+s}{\PYGZdq{}}\PYG{l+s}{Tampere Streetlights}\PYG{l+s}{\PYGZdq{}}
    \PYG{p}{\PYGZcb{}}
\PYG{p}{]}
\end{Verbatim}

\begin{notice}{note}{Note:}
The name element is the one to be added to the url.
\end{notice}


\subsection{/locationdata/api/\textless{}collection name\textgreater{}/}
\label{restdoc:locationdata-api-collection-name}
\textbf{Allowed methods:}
\begin{itemize}
\item {} 
GET
\begin{itemize}
\item {} 
Returns the whole collection.

\end{itemize}

\item {} 
DELETE
\begin{itemize}
\item {} 
Deletes the whole collection.

\end{itemize}

\item {} 
PUT
\begin{itemize}
\item {} 
Replaces the collection.

\end{itemize}

\item {} 
POST
\begin{itemize}
\item {} 
Adds a new element to the collection.

\end{itemize}

\end{itemize}

\textbf{URL parameters}
\begin{itemize}
\item {} 
mini (\emph{Optional})
\begin{itemize}
\item {} 
\textbf{Boolean} Returns minimum amount of data. Valid values: true or false

\end{itemize}

\end{itemize}


\subsection{/locationdata/api/\textless{}collection name\textgreater{}/\textless{}resource\textgreater{}}
\label{restdoc:locationdata-api-collection-name-resource}
\textbf{Allowed methods:}
\begin{itemize}
\item {} 
GET
\begin{itemize}
\item {} 
Returns the resource.

\end{itemize}

\item {} 
DELETE
\begin{itemize}
\item {} 
Deletes the resource.

\end{itemize}

\item {} 
PUT
\begin{itemize}
\item {} 
Update or create a resource.

\end{itemize}

\end{itemize}


\subsection{/locationdata/api/\textless{}collection name\textgreater{}/near/}
\label{restdoc:locationdata-api-collection-name-near}
Searches objects from circular area.

\textbf{Allowed methods:}
\begin{itemize}
\item {} 
GET
\begin{itemize}
\item {} 
Returns the resources near the location.

\end{itemize}

\item {} 
DELETE
\begin{itemize}
\item {} 
Deletes the resources near the location.

\end{itemize}

\end{itemize}

\textbf{URL parameters}
\begin{itemize}
\item {} 
mini (\emph{Optional})
\begin{itemize}
\item {} 
\textbf{Boolean} Returns minimum amount of data. Valid values: true or false

\end{itemize}

\item {} 
latitude (\emph{Required})
\begin{itemize}
\item {} 
\textbf{Float} The latitude of the circle's center

\end{itemize}

\item {} 
longitude (\emph{Required})
\begin{itemize}
\item {} 
\textbf{Float} The longitude of the circle's center

\end{itemize}

\item {} 
range (\emph{Optional})
\begin{itemize}
\item {} 
\textbf{Float} The radius of the circle

\end{itemize}

\end{itemize}


\subsection{/locationdata/api/\textless{}collection name\textgreater{}/inarea/}
\label{restdoc:locationdata-api-collection-name-inarea}
Searches objects inside a rectangular area.

\textbf{Allowed methods:}
\begin{itemize}
\item {} 
GET
\begin{itemize}
\item {} 
Returns the resources inside the area.

\end{itemize}

\item {} 
DELETE
\begin{itemize}
\item {} 
Deletes the resource inside the area.

\end{itemize}

\end{itemize}

\textbf{URL parameters}
\begin{itemize}
\item {} 
mini (\emph{Optional})
\begin{itemize}
\item {} 
\textbf{Boolean} Returns minimum amount of data. Valid values: true or false

\end{itemize}

\item {} 
xbottomleft (\emph{Required})
\begin{itemize}
\item {} 
\textbf{Float} The longitude of the bottom left corner of the area.

\end{itemize}

\item {} 
ybottomleft (\emph{Required})
\begin{itemize}
\item {} 
\textbf{Float} The latitude of the bottom left corner of the area.

\end{itemize}

\item {} 
xtopright (\emph{Required})
\begin{itemize}
\item {} 
\textbf{Float} The longitude of the top right corner of the area.

\end{itemize}

\item {} 
ytopright (\emph{Required})
\begin{itemize}
\item {} 
\textbf{Float} The latitude of the top right corner of the area.

\end{itemize}

\end{itemize}


\subsection{/locationdata/api/\textless{}collection name\textgreater{}/search/}
\label{restdoc:locationdata-api-collection-name-search}
Searches from the location data REST. Search is currently limited to the id.

\textbf{Allowed methods:}
\begin{itemize}
\item {} 
POST
\begin{itemize}
\item {} 
Send the search JSON.

\end{itemize}

\end{itemize}

\textbf{URL parameters}
\begin{itemize}
\item {} 
None

\end{itemize}


\section{Message data REST}
\label{restdoc:messagerest}\label{restdoc:message-data-rest}
The REST for sending messages in the system. For JSON formats, see {\hyperref[json/messagejson:messagejson]{\emph{Message formats}}}

In URLs \textbf{\textless{}message id\textgreater{}} and \textbf{\textless{}category\textgreater{}} are to be replaced with appropriate values. Message id is an integer and
category is a string.


\subsection{/messagedata/api/send/}
\label{restdoc:messagedata-api-send}
\textbf{Allowed methods:}
\begin{itemize}
\item {} 
POST
\begin{itemize}
\item {} 
Send a message.

\end{itemize}

\end{itemize}

\textbf{URL parameters}
\begin{itemize}
\item {} 
None

\end{itemize}


\subsection{/messagedata/api/users/list/}
\label{restdoc:messagedata-api-users-list}
Lists all users.

\textbf{Allowed methods:}
\begin{itemize}
\item {} 
GET
\begin{itemize}
\item {} 
Returns name and email of users.

\end{itemize}

\end{itemize}

\textbf{URL parameters}
\begin{itemize}
\item {} 
None

\end{itemize}


\subsection{/messagedata/api/markasread/\textless{}message id\textgreater{}}
\label{restdoc:messagedata-api-markasread-message-id}
\textbf{Allowed methods:}
\begin{itemize}
\item {} 
GET
\begin{itemize}
\item {} 
Mark message read.

\end{itemize}

\end{itemize}

\textbf{URL parameters}
\begin{itemize}
\item {} 
None

\end{itemize}


\subsection{/messagedata/api/messages/}
\label{restdoc:messagedata-api-messages}
\textbf{Allowed methods:}
\begin{itemize}
\item {} 
GET
\begin{itemize}
\item {} 
Get user's all messages.

\end{itemize}

\item {} 
DELETE
\begin{itemize}
\item {} 
Delete user's all messages

\end{itemize}

\end{itemize}

\textbf{URL parameters}
\begin{itemize}
\item {} 
None

\end{itemize}


\subsection{/messagedata/api/messages/\textless{}message id\textgreater{}}
\label{restdoc:messagedata-api-messages-message-id}
\textbf{Allowed methods:}
\begin{itemize}
\item {} 
GET
\begin{itemize}
\item {} 
Get a single message.

\end{itemize}

\item {} 
DELETE
\begin{itemize}
\item {} 
Delete the message.

\end{itemize}

\end{itemize}

\textbf{URL parameters}
\begin{itemize}
\item {} 
None

\end{itemize}


\subsection{/messagedata/api/messages/\textless{}category\textgreater{}/}
\label{restdoc:messagedata-api-messages-category}
\textbf{Allowed methods:}
\begin{itemize}
\item {} 
GET
\begin{itemize}
\item {} 
Get user's all messages in certain category.

\end{itemize}

\item {} 
DELETE
\begin{itemize}
\item {} 
Delete user's all messages in certain category.

\end{itemize}

\end{itemize}

\textbf{URL parameters}
\begin{itemize}
\item {} 
None

\end{itemize}


\subsection{/messagedata/api/messages/\textless{}category\textgreater{}/\textless{}message id\textgreater{}}
\label{restdoc:messagedata-api-messages-category-message-id}
\textbf{Allowed methods:}
\begin{itemize}
\item {} 
GET
\begin{itemize}
\item {} 
Get a single message in a certain category.

\end{itemize}

\item {} 
DELETE
\begin{itemize}
\item {} 
Delete a single message in a certain category.

\end{itemize}

\end{itemize}

\textbf{URL parameters}
\begin{itemize}
\item {} 
None

\end{itemize}


\chapter{Code documentation: REST}
\label{codedoc:code-documentation-rest}\label{codedoc::doc}

\section{\texttt{lbd\_backend.LBD\_REST\_locationdata.decorators} Location data decorators}
\label{codedoc/decdoc:module-lbd_backend.LBD_REST_locationdata.decorators}\label{codedoc/decdoc:lbd-backend-lbd-rest-locationdata-decorators-location-data-decorators}\label{codedoc/decdoc::doc}\index{lbd\_backend.LBD\_REST\_locationdata.decorators (module)}

\subsection{Decorators for location data REST}
\label{codedoc/decdoc:locdecos}\label{codedoc/decdoc:decorators-for-location-data-rest}
\textbf{This module contains the decorators for the REST handling the location data}
\phantomsection\label{codedoc/decdoc:module-LocationdataREST.decorators}\index{LocationdataREST.decorators (module)}\index{lbd\_require\_login() (in module lbd\_backend.LBD\_REST\_locationdata.decorators)}

\begin{fulllineitems}
\phantomsection\label{codedoc/decdoc:lbd_backend.LBD_REST_locationdata.decorators.lbd_require_login}\pysiglinewithargsret{\code{lbd\_backend.LBD\_REST\_locationdata.decorators.}\bfcode{lbd\_require\_login}}{\emph{func}}{}
\emph{Wrapper}

This wrapper is used for authenticating the user with Google OAuth2.

Key ``lbduser'' is added to kwargs with User object as value.

\end{fulllineitems}

\index{location\_collection() (in module lbd\_backend.LBD\_REST\_locationdata.decorators)}

\begin{fulllineitems}
\phantomsection\label{codedoc/decdoc:lbd_backend.LBD_REST_locationdata.decorators.location_collection}\pysiglinewithargsret{\code{lbd\_backend.LBD\_REST\_locationdata.decorators.}\bfcode{location\_collection}}{\emph{func}}{}
\emph{Wrapper}

Checks if the collection in the URL exists and the handler for it is installed.

Key ``handlerinterface'' is added to kwargs with a handler object as the value.

\end{fulllineitems}



\section{\texttt{lbd\_backend.LBD\_REST\_locationdata} Location data REST}
\label{codedoc/locdoc:lbd-backend-lbd-rest-locationdata-location-data-rest}\label{codedoc/locdoc::doc}\label{codedoc/locdoc:module-lbd_backend.LBD_REST_locationdata.views}\index{lbd\_backend.LBD\_REST\_locationdata.views (module)}

\subsection{View for handling the backend REST locationdata requests}
\label{codedoc/locdoc:view-for-handling-the-backend-rest-locationdata-requests}\label{codedoc/locdoc:module-LocationdataREST.views}\index{LocationdataREST.views (module)}
\textbf{This module handles http requests related to location data.}

For all possible HTTP statuses, see {\hyperref[restdoc:restdoc]{\emph{REST documentation}}}.

Status 200 is returned when request is valid and handled successfully while 400 is returned when the request conten
is malformed or there is some other issues with the request..

\begin{notice}{note}{Note:}
In case of PUT and POST, status 200 does not guarantee that any data has changed in database.
\end{notice}

Status 400 is returned when request body does not match the defined format or there is some other inconsistency in
the request.

Status 500 means that something went wrong when handling the request.

Client should be able to handle these responses and should not crash in case some undefined status is returned for
reasons unknown.

\begin{notice}{note}{Note:}
For kwargs added by the decorators, see {\hyperref[codedoc/decdoc:locdecos]{\emph{the decorator documentation}}}.
\end{notice}
\index{api() (in module lbd\_backend.LBD\_REST\_locationdata.views)}

\begin{fulllineitems}
\phantomsection\label{codedoc/locdoc:lbd_backend.LBD_REST_locationdata.views.api}\pysiglinewithargsret{\code{lbd\_backend.LBD\_REST\_locationdata.views.}\bfcode{api}}{\emph{request}, \emph{*args}, \emph{**kwargs}}{}
This view returns the installed open data sources as JSON.

\textbf{Supported HTTP methods:}
\begin{itemize}
\item {} 
GET

\end{itemize}
\begin{quote}\begin{description}
\item[{Returns}] \leavevmode
HTTP response.

\end{description}\end{quote}

\end{fulllineitems}

\index{collection() (in module lbd\_backend.LBD\_REST\_locationdata.views)}

\begin{fulllineitems}
\phantomsection\label{codedoc/locdoc:lbd_backend.LBD_REST_locationdata.views.collection}\pysiglinewithargsret{\code{lbd\_backend.LBD\_REST\_locationdata.views.}\bfcode{collection}}{\emph{request}, \emph{*args}, \emph{**kwargs}}{}
REST main collection request handler.

\textbf{Supported HTTP methods:}
\begin{itemize}
\item {} 
GET

\item {} 
DELETE

\item {} 
PUT

\item {} 
POST

\end{itemize}
\begin{quote}\begin{description}
\item[{Parameters}] \leavevmode\begin{itemize}
\item {} 
\textbf{request} -- Request object

\item {} 
\textbf{args} -- arguments

\item {} 
\textbf{kwargs} -- Dictionary (keyword arguments). Known kwargs listed below.

\end{itemize}

\end{description}\end{quote}

\textbf{In addition to the kwargs added by the decorators, this view uses the following:}
\begin{itemize}
\item {} 
collection (String)
\begin{itemize}
\item {} 
Location data collection name

\end{itemize}

\end{itemize}

\textbf{Supported URL parameter:}
\begin{itemize}
\item {} 
mini (True or False): Return minimum amount of data (response must still be valid GeoJSON

\end{itemize}

\end{fulllineitems}

\index{collection\_inarea() (in module lbd\_backend.LBD\_REST\_locationdata.views)}

\begin{fulllineitems}
\phantomsection\label{codedoc/locdoc:lbd_backend.LBD_REST_locationdata.views.collection_inarea}\pysiglinewithargsret{\code{lbd\_backend.LBD\_REST\_locationdata.views.}\bfcode{collection\_inarea}}{\emph{request}, \emph{*args}, \emph{**kwargs}}{}
REST subcollection ``inarea'' request handler. Handles objects inside a rectangular area.

\textbf{Supported HTTP methods:}
\begin{itemize}
\item {} 
GET

\item {} 
DELETE

\end{itemize}
\begin{quote}\begin{description}
\item[{Parameters}] \leavevmode\begin{itemize}
\item {} 
\textbf{request} -- Request object

\item {} 
\textbf{args} -- arguments

\item {} 
\textbf{kwargs} -- Dictionary (keyword arguments). Known kwargs listed below.

\end{itemize}

\end{description}\end{quote}

\textbf{In addition to the kwargs added by the decorators, this view uses the following:}
\begin{itemize}
\item {} 
collection (String)
\begin{itemize}
\item {} 
Location data collection name

\end{itemize}

\end{itemize}

\textbf{Supported URL parameter:}
\begin{itemize}
\item {} 
xbottomleft (Float): The x-coordinate of the bottom left corner of the area

\item {} 
ybottomleft (Float): The y-coordinate of the bottom left corner of the area

\item {} 
xtopright (Float): The x-coordinate of the top right corner of the area

\item {} 
ytopright (Float): The y-coordinate of the top right corner of the area

\item {} 
mini (True or False): Return minimum amount of data (response must still be valid GeoJSON

\end{itemize}

\end{fulllineitems}

\index{collection\_near() (in module lbd\_backend.LBD\_REST\_locationdata.views)}

\begin{fulllineitems}
\phantomsection\label{codedoc/locdoc:lbd_backend.LBD_REST_locationdata.views.collection_near}\pysiglinewithargsret{\code{lbd\_backend.LBD\_REST\_locationdata.views.}\bfcode{collection\_near}}{\emph{request}, \emph{*args}, \emph{**kwargs}}{}
REST subcollection ``near'' request handler. Handles objects in certain range of given coordinates

\textbf{Supported HTTP methods:}
\begin{itemize}
\item {} 
GET

\item {} 
DELETE

\end{itemize}
\begin{quote}\begin{description}
\item[{Parameters}] \leavevmode\begin{itemize}
\item {} 
\textbf{request} -- Request object

\item {} 
\textbf{args} -- arguments

\item {} 
\textbf{kwargs} -- Dictionary (keyword arguments). Known kwargs listed below.

\end{itemize}

\end{description}\end{quote}

\textbf{In addition to the kwargs added by the decorators, this view uses the following:}
\begin{itemize}
\item {} 
collection (String)
\begin{itemize}
\item {} 
Location data collection name

\end{itemize}

\end{itemize}

\textbf{Supported URL parameter:}
\begin{itemize}
\item {} 
latitude (Float): the latitude of the center \textbf{REQUIRED}

\item {} 
longitude (Float): the longitude of the center \textbf{REQUIRED}

\item {} 
range (Float): the radius of the area

\item {} 
mini (True or False): Return minimum amount of data (response must still be valid GeoJSON

\end{itemize}

\end{fulllineitems}

\index{search\_from\_rest() (in module lbd\_backend.LBD\_REST\_locationdata.views)}

\begin{fulllineitems}
\phantomsection\label{codedoc/locdoc:lbd_backend.LBD_REST_locationdata.views.search_from_rest}\pysiglinewithargsret{\code{lbd\_backend.LBD\_REST\_locationdata.views.}\bfcode{search\_from\_rest}}{\emph{request}, \emph{*args}, \emph{**kwargs}}{}
This view searches for the given search phrase from the database. Currently only search from id field is supported.
For json format, see {\hyperref[json/search:searchjson]{\emph{Search}}}.

\textbf{Supported HTTP methods:}
\begin{itemize}
\item {} 
POST

\end{itemize}
\begin{quote}\begin{description}
\item[{Parameters}] \leavevmode\begin{itemize}
\item {} 
\textbf{request} -- Request object

\item {} 
\textbf{args} -- Arguments

\item {} 
\textbf{kwargs} -- Keyword arguments

\end{itemize}

\item[{Returns}] \leavevmode
HTTP response

\end{description}\end{quote}

\end{fulllineitems}

\index{single\_resource() (in module lbd\_backend.LBD\_REST\_locationdata.views)}

\begin{fulllineitems}
\phantomsection\label{codedoc/locdoc:lbd_backend.LBD_REST_locationdata.views.single_resource}\pysiglinewithargsret{\code{lbd\_backend.LBD\_REST\_locationdata.views.}\bfcode{single\_resource}}{\emph{request}, \emph{*args}, \emph{**kwargs}}{}
REST single resource (in certain collection) request handler.

\textbf{Supported HTTP methods:}
\begin{itemize}
\item {} 
GET

\item {} 
DELETE

\item {} 
PUT

\end{itemize}
\begin{quote}\begin{description}
\item[{Parameters}] \leavevmode\begin{itemize}
\item {} 
\textbf{request} -- Request object

\item {} 
\textbf{args} -- arguments

\item {} 
\textbf{kwargs} -- Dictionary (keyword arguments). Known kwargs listed below.

\end{itemize}

\end{description}\end{quote}

\textbf{In addition to the kwargs added by the decorators, this view uses the following:}
\begin{itemize}
\item {} 
collection (String)
\begin{itemize}
\item {} 
Location data collection name

\end{itemize}

\item {} 
resource (String)
\begin{itemize}
\item {} 
Resource id

\end{itemize}

\end{itemize}
\begin{quote}\begin{description}
\item[{Returns}] \leavevmode
HTTP response. Possible statuses are listed in module documentation

\end{description}\end{quote}

\end{fulllineitems}

\phantomsection\label{codedoc/locdoc:module-lbd_backend.LBD_REST_locationdata.models}\index{lbd\_backend.LBD\_REST\_locationdata.models (module)}

\subsection{Model containing the metadata database structure}
\label{codedoc/locdoc:module-LocationdataREST.models}\label{codedoc/locdoc:model-containing-the-metadata-database-structure}\index{LocationdataREST.models (module)}\index{MetaData (class in lbd\_backend.LBD\_REST\_locationdata.models)}

\begin{fulllineitems}
\phantomsection\label{codedoc/locdoc:lbd_backend.LBD_REST_locationdata.models.MetaData}\pysiglinewithargsret{\strong{class }\code{lbd\_backend.LBD\_REST\_locationdata.models.}\bfcode{MetaData}}{\emph{*args}, \emph{**kwargs}}{}
Fields:
\begin{itemize}
\item {} 
status: StringField \textbf{REQUIRED}

\item {} 
modified: IntField \textbf{REQUIRED}

\item {} 
modifier: IntField \textbf{REQUIRED}

\item {} 
info: StringField \textbf{REQUIRED}

\end{itemize}

\textbf{Status} is a string describing the status of the object. It is always required if metadata for the
object is defined.

\textbf{Modified} is a timestamp (seconds from epoch) and is generated automatically by the system. Always required.

\textbf{Modifier} is the id of the user that modified the metadata item. Always required, inserted by the system.

\textbf{Info} is ... infofield?

New fields can be dynamically added into this model.

\end{fulllineitems}

\index{MetaDocument (class in lbd\_backend.LBD\_REST\_locationdata.models)}

\begin{fulllineitems}
\phantomsection\label{codedoc/locdoc:lbd_backend.LBD_REST_locationdata.models.MetaDocument}\pysiglinewithargsret{\strong{class }\code{lbd\_backend.LBD\_REST\_locationdata.models.}\bfcode{MetaDocument}}{\emph{*args}, \emph{**values}}{}
Fields
\begin{itemize}
\item {} 
feature\_id: StringField \textbf{REQUIRED} \textbf{UNIQUE}

\item {} 
collection: StringField \textbf{REQUIRED}

\item {} 
meta\_data: EmbeddeDocumentField(MetaData) \textbf{REQUIRED}

\end{itemize}

\textbf{Feature\_id} is a string that combines the metadata to an object. Simulates a foreign key.

\textbf{Collection\_id} is a string that tells the collection where the metadata belongs.

\textbf{Meta\_data} is an embedded document.

New fields can be dynamically added into this model.

\end{fulllineitems}



\section{\texttt{lbd\_backend.LBD\_REST\_messagedata} Message data REST}
\label{codedoc/msgdoc:module-lbd_backend.LBD_REST_messagedata.views}\label{codedoc/msgdoc::doc}\label{codedoc/msgdoc:lbd-backend-lbd-rest-messagedata-message-data-rest}\index{lbd\_backend.LBD\_REST\_messagedata.views (module)}

\subsection{View for handling messages}
\label{codedoc/msgdoc:module-MessagedataREST.views}\label{codedoc/msgdoc:view-for-handling-messages}\index{MessagedataREST.views (module)}\index{mark\_as\_read() (in module lbd\_backend.LBD\_REST\_messagedata.views)}

\begin{fulllineitems}
\phantomsection\label{codedoc/msgdoc:lbd_backend.LBD_REST_messagedata.views.mark_as_read}\pysiglinewithargsret{\code{lbd\_backend.LBD\_REST\_messagedata.views.}\bfcode{mark\_as\_read}}{\emph{request}, \emph{*args}, \emph{**kwargs}}{}
View for marking a message read.

\textbf{Supported HTTP methods:}
\begin{itemize}
\item {} 
GET

\end{itemize}
\begin{quote}\begin{description}
\item[{Parameters}] \leavevmode\begin{itemize}
\item {} 
\textbf{request} -- Request object

\item {} 
\textbf{args} -- arguments

\item {} 
\textbf{kwargs} -- Dictionary (keyword arguments). Known kwargs listed below.

\end{itemize}

\end{description}\end{quote}

\textbf{The method uses the following kwargs:}
\begin{itemize}
\item {} 
message (Integer)

\item {} 
lbduser (User)

\end{itemize}

\emph{Message} specifies the message id. Required.
\begin{quote}\begin{description}
\item[{Returns}] \leavevmode
HTTP response. Possible statuses are listed in module documentation

\end{description}\end{quote}

\end{fulllineitems}

\index{msg\_general() (in module lbd\_backend.LBD\_REST\_messagedata.views)}

\begin{fulllineitems}
\phantomsection\label{codedoc/msgdoc:lbd_backend.LBD_REST_messagedata.views.msg_general}\pysiglinewithargsret{\code{lbd\_backend.LBD\_REST\_messagedata.views.}\bfcode{msg\_general}}{\emph{request}, \emph{*args}, \emph{**kwargs}}{}
Handles all message requests (both to single and multiple messages).

\textbf{Supported HTTP methods:}
\begin{itemize}
\item {} 
GET

\item {} 
DELETE

\end{itemize}
\begin{quote}\begin{description}
\item[{Parameters}] \leavevmode\begin{itemize}
\item {} 
\textbf{request} -- Request object

\item {} 
\textbf{args} -- arguments

\item {} 
\textbf{kwargs} -- Dictionary (keyword arguments). Known kwargs listed below.

\end{itemize}

\end{description}\end{quote}

\textbf{The method uses the following kwargs:}
\begin{itemize}
\item {} 
category (String)

\item {} 
message (Integer)

\item {} 
lbduser (User)

\end{itemize}

\emph{Category} specifies the message category. This argument is used only if it is specified in the url. Category
is equivalent to locationdata collection. If this argument is used, it is expected that a locationdata collection
with the same name exists and is ``installed''.

\emph{Message} specifies the message id. Used only if specified in the url.
\begin{quote}\begin{description}
\item[{Returns}] \leavevmode
HTTP response. Possible statuses are listed in module documentation

\end{description}\end{quote}

\end{fulllineitems}

\index{msg\_send() (in module lbd\_backend.LBD\_REST\_messagedata.views)}

\begin{fulllineitems}
\phantomsection\label{codedoc/msgdoc:lbd_backend.LBD_REST_messagedata.views.msg_send}\pysiglinewithargsret{\code{lbd\_backend.LBD\_REST\_messagedata.views.}\bfcode{msg\_send}}{\emph{request}, \emph{*args}, \emph{**kwargs}}{}
View for sending messages.

\textbf{Supported HTTP methods:}
\begin{itemize}
\item {} 
POST

\end{itemize}
\begin{quote}\begin{description}
\item[{Parameters}] \leavevmode\begin{itemize}
\item {} 
\textbf{request} -- Request object

\item {} 
\textbf{args} -- arguments

\item {} 
\textbf{kwargs} -- Dictionary (keyword arguments). Known kwargs listed below.

\end{itemize}

\end{description}\end{quote}

\textbf{The method uses the following kwargs:}
\begin{itemize}
\item {} 
lbduser (User)

\end{itemize}
\begin{quote}\begin{description}
\item[{Returns}] \leavevmode
HTTP response. Possible statuses are listed in module documentation

\end{description}\end{quote}

\end{fulllineitems}

\phantomsection\label{codedoc/msgdoc:module-lbd_backend.LBD_REST_messagedata.models}\index{lbd\_backend.LBD\_REST\_messagedata.models (module)}

\subsection{Model for message}
\label{codedoc/msgdoc:module-MessagedataREST.models}\label{codedoc/msgdoc:model-for-message}\index{MessagedataREST.models (module)}\index{Attachment (class in lbd\_backend.LBD\_REST\_messagedata.models)}

\begin{fulllineitems}
\phantomsection\label{codedoc/msgdoc:lbd_backend.LBD_REST_messagedata.models.Attachment}\pysiglinewithargsret{\strong{class }\code{lbd\_backend.LBD\_REST\_messagedata.models.}\bfcode{Attachment}}{\emph{*args}, \emph{**kwargs}}{}
Fields:
\begin{itemize}
\item {} 
category: StringField \textbf{REQUIRED}

\item {} 
aid: StringField \textbf{REQUIRED}

\end{itemize}

\textbf{category} is the name of the locationdata collection to which the attachment refers.

\textbf{aid} is the id (not mongodb id, but the back-end's id) of the object to which the attachment refers (``foreignkey'').

\end{fulllineitems}

\index{Message (class in lbd\_backend.LBD\_REST\_messagedata.models)}

\begin{fulllineitems}
\phantomsection\label{codedoc/msgdoc:lbd_backend.LBD_REST_messagedata.models.Message}\pysiglinewithargsret{\strong{class }\code{lbd\_backend.LBD\_REST\_messagedata.models.}\bfcode{Message}}{\emph{*args}, \emph{**values}}{}
Fields:
\begin{itemize}
\item {} 
mid: SequenceField \textbf{AUTOMATIC}

\item {} 
category: StringField

\item {} 
sender: EmailField \textbf{REQUIRED}

\item {} 
recipient: EmailField \textbf{REQUIRED}

\item {} 
attachments: ListField(EmbeddedDocumentField(Attachment))

\item {} 
topic: StringField \textbf{REQUIRED}

\item {} 
message: StringField \textbf{REQUIRED}

\item {} 
messageread: BooleanField \textbf{REQUIRED}

\item {} 
timestamp: IntField \textbf{REQUIRED}

\end{itemize}

\textbf{Mid} is the id of the message. Generated automatically.

\textbf{Category} is the name of the locationdata collection to which the message refers.

\textbf{Sender} is the email address of the sender.

\textbf{Recipient} is the email address of the recipient.

\textbf{Attachment} is a list of Attachment objects.

\textbf{Topic} is the topic of the message.

\textbf{Message} is the message content.

\textbf{Messageread} tells if the message has been read or not. (True or False) (False by default)

\textbf{Timestamp} tells when the message was sent. Timestamp is in seconds from Unix Epoch on January 1st, 1970 at UTC.

\end{fulllineitems}



\chapter{Code documentation: Handlers}
\label{hand:code-documentation-handlers}\label{hand::doc}

\section{\texttt{RESThandlers.HandlerInterface} REST Handler Interface}
\label{handler/interface:resthandlers-handlerinterface-rest-handler-interface}\label{handler/interface::doc}\label{handler/interface:module-RESThandlers.HandlerInterface.Exceptions}\index{RESThandlers.HandlerInterface.Exceptions (module)}\phantomsection\label{handler/interface:module-Handlers.Interface.Exceptions}\index{Handlers.Interface.Exceptions (module)}\index{CollectionNotInstalled}

\begin{fulllineitems}
\phantomsection\label{handler/interface:RESThandlers.HandlerInterface.Exceptions.CollectionNotInstalled}\pysigline{\strong{exception }\code{RESThandlers.HandlerInterface.Exceptions.}\bfcode{CollectionNotInstalled}}
Bases: \code{exceptions.Exception}

\end{fulllineitems}

\index{GenericDBError}

\begin{fulllineitems}
\phantomsection\label{handler/interface:RESThandlers.HandlerInterface.Exceptions.GenericDBError}\pysigline{\strong{exception }\code{RESThandlers.HandlerInterface.Exceptions.}\bfcode{GenericDBError}}
Bases: \code{exceptions.Exception}

\end{fulllineitems}

\index{MultipleObjectsFound}

\begin{fulllineitems}
\phantomsection\label{handler/interface:RESThandlers.HandlerInterface.Exceptions.MultipleObjectsFound}\pysigline{\strong{exception }\code{RESThandlers.HandlerInterface.Exceptions.}\bfcode{MultipleObjectsFound}}
Bases: \code{exceptions.Exception}

\end{fulllineitems}

\index{ObjectNotFound}

\begin{fulllineitems}
\phantomsection\label{handler/interface:RESThandlers.HandlerInterface.Exceptions.ObjectNotFound}\pysigline{\strong{exception }\code{RESThandlers.HandlerInterface.Exceptions.}\bfcode{ObjectNotFound}}
Bases: \code{exceptions.Exception}

\end{fulllineitems}

\phantomsection\label{handler/interface:module-RESThandlers.HandlerInterface.HandlerBaseClass}\index{RESThandlers.HandlerInterface.HandlerBaseClass (module)}\phantomsection\label{handler/interface:module-Handlers.Interface.base}\index{Handlers.Interface.base (module)}\index{HandlerBase (class in RESThandlers.HandlerInterface.HandlerBaseClass)}

\begin{fulllineitems}
\phantomsection\label{handler/interface:RESThandlers.HandlerInterface.HandlerBaseClass.HandlerBase}\pysigline{\strong{class }\code{RESThandlers.HandlerInterface.HandlerBaseClass.}\bfcode{HandlerBase}}
Bases: \code{object}
\index{delete\_all() (RESThandlers.HandlerInterface.HandlerBaseClass.HandlerBase method)}

\begin{fulllineitems}
\phantomsection\label{handler/interface:RESThandlers.HandlerInterface.HandlerBaseClass.HandlerBase.delete_all}\pysiglinewithargsret{\bfcode{delete\_all}}{}{}
\end{fulllineitems}

\index{delete\_item\_by\_id() (RESThandlers.HandlerInterface.HandlerBaseClass.HandlerBase method)}

\begin{fulllineitems}
\phantomsection\label{handler/interface:RESThandlers.HandlerInterface.HandlerBaseClass.HandlerBase.delete_item_by_id}\pysiglinewithargsret{\bfcode{delete\_item\_by\_id}}{\emph{iid}}{}
\end{fulllineitems}

\index{delete\_near() (RESThandlers.HandlerInterface.HandlerBaseClass.HandlerBase method)}

\begin{fulllineitems}
\phantomsection\label{handler/interface:RESThandlers.HandlerInterface.HandlerBaseClass.HandlerBase.delete_near}\pysiglinewithargsret{\bfcode{delete\_near}}{\emph{latitude}, \emph{longitude}, \emph{nrange}}{}
\end{fulllineitems}

\index{get\_all() (RESThandlers.HandlerInterface.HandlerBaseClass.HandlerBase method)}

\begin{fulllineitems}
\phantomsection\label{handler/interface:RESThandlers.HandlerInterface.HandlerBaseClass.HandlerBase.get_all}\pysiglinewithargsret{\bfcode{get\_all}}{\emph{mini=False}}{}
\end{fulllineitems}

\index{get\_by\_id() (RESThandlers.HandlerInterface.HandlerBaseClass.HandlerBase method)}

\begin{fulllineitems}
\phantomsection\label{handler/interface:RESThandlers.HandlerInterface.HandlerBaseClass.HandlerBase.get_by_id}\pysiglinewithargsret{\bfcode{get\_by\_id}}{\emph{iid}}{}
\end{fulllineitems}

\index{get\_field\_names() (RESThandlers.HandlerInterface.HandlerBaseClass.HandlerBase method)}

\begin{fulllineitems}
\phantomsection\label{handler/interface:RESThandlers.HandlerInterface.HandlerBaseClass.HandlerBase.get_field_names}\pysiglinewithargsret{\bfcode{get\_field\_names}}{}{}
\end{fulllineitems}

\index{get\_item\_count() (RESThandlers.HandlerInterface.HandlerBaseClass.HandlerBase method)}

\begin{fulllineitems}
\phantomsection\label{handler/interface:RESThandlers.HandlerInterface.HandlerBaseClass.HandlerBase.get_item_count}\pysiglinewithargsret{\bfcode{get\_item\_count}}{}{}
\end{fulllineitems}

\index{get\_near() (RESThandlers.HandlerInterface.HandlerBaseClass.HandlerBase method)}

\begin{fulllineitems}
\phantomsection\label{handler/interface:RESThandlers.HandlerInterface.HandlerBaseClass.HandlerBase.get_near}\pysiglinewithargsret{\bfcode{get\_near}}{\emph{longitude}, \emph{latitude}, \emph{nrange}}{}
\end{fulllineitems}

\index{get\_within\_rectangle() (RESThandlers.HandlerInterface.HandlerBaseClass.HandlerBase method)}

\begin{fulllineitems}
\phantomsection\label{handler/interface:RESThandlers.HandlerInterface.HandlerBaseClass.HandlerBase.get_within_rectangle}\pysiglinewithargsret{\bfcode{get\_within\_rectangle}}{\emph{xtop\_right}, \emph{ytop\_right}, \emph{xbottom\_left}, \emph{ybottom\_left}, \emph{mini=False}}{}
\end{fulllineitems}

\index{handler\_id (RESThandlers.HandlerInterface.HandlerBaseClass.HandlerBase attribute)}

\begin{fulllineitems}
\phantomsection\label{handler/interface:RESThandlers.HandlerInterface.HandlerBaseClass.HandlerBase.handler_id}\pysigline{\bfcode{handler\_id}}
staticmethod(function) -\textgreater{} method

Convert a function to be a static method.

A static method does not receive an implicit first argument.
To declare a static method, use this idiom:
\begin{quote}

class C:
def f(arg1, arg2, ...): ...
f = staticmethod(f)
\end{quote}

It can be called either on the class (e.g. C.f()) or on an instance
(e.g. C().f()).  The instance is ignored except for its class.

Static methods in Python are similar to those found in Java or C++.
For a more advanced concept, see the classmethod builtin.

\end{fulllineitems}

\index{insert\_to\_db() (RESThandlers.HandlerInterface.HandlerBaseClass.HandlerBase method)}

\begin{fulllineitems}
\phantomsection\label{handler/interface:RESThandlers.HandlerInterface.HandlerBaseClass.HandlerBase.insert_to_db}\pysiglinewithargsret{\bfcode{insert\_to\_db}}{\emph{jsonitem}}{}
\end{fulllineitems}

\index{search() (RESThandlers.HandlerInterface.HandlerBaseClass.HandlerBase method)}

\begin{fulllineitems}
\phantomsection\label{handler/interface:RESThandlers.HandlerInterface.HandlerBaseClass.HandlerBase.search}\pysiglinewithargsret{\bfcode{search}}{\emph{phrase}, \emph{field}}{}
\end{fulllineitems}

\index{update\_db() (RESThandlers.HandlerInterface.HandlerBaseClass.HandlerBase method)}

\begin{fulllineitems}
\phantomsection\label{handler/interface:RESThandlers.HandlerInterface.HandlerBaseClass.HandlerBase.update_db}\pysiglinewithargsret{\bfcode{update\_db}}{}{}
\end{fulllineitems}


\end{fulllineitems}



\section{\texttt{RESThandlers.HandlerInterface.Factory} Handler Factory}
\label{handler/factory::doc}\label{handler/factory:resthandlers-handlerinterface-factory-handler-factory}\label{handler/factory:module-RESThandlers.HandlerInterface.Factory}\index{RESThandlers.HandlerInterface.Factory (module)}\phantomsection\label{handler/factory:module-Handlers.Interface.Factory}\index{Handlers.Interface.Factory (module)}\index{HandlerFactory (class in RESThandlers.HandlerInterface.Factory)}

\begin{fulllineitems}
\phantomsection\label{handler/factory:RESThandlers.HandlerInterface.Factory.HandlerFactory}\pysiglinewithargsret{\strong{class }\code{RESThandlers.HandlerInterface.Factory.}\bfcode{HandlerFactory}}{\emph{collection}}{}
Bases: \code{object}
\index{create() (RESThandlers.HandlerInterface.Factory.HandlerFactory method)}

\begin{fulllineitems}
\phantomsection\label{handler/factory:RESThandlers.HandlerInterface.Factory.HandlerFactory.create}\pysiglinewithargsret{\bfcode{create}}{}{}
Creates and return a handler object.
\begin{quote}\begin{description}
\item[{Returns}] \leavevmode
Handler object

\end{description}\end{quote}

\end{fulllineitems}

\index{get\_installed() (RESThandlers.HandlerInterface.Factory.HandlerFactory static method)}

\begin{fulllineitems}
\phantomsection\label{handler/factory:RESThandlers.HandlerInterface.Factory.HandlerFactory.get_installed}\pysiglinewithargsret{\strong{static }\bfcode{get\_installed}}{}{}
\textbf{Static}

This method returns the installed open data services as dictionary where ``name'' is the name of the service (used
when creating a new handler object) and ``fields'' tells what elements the service provides.
:return: dictionary

\end{fulllineitems}


\end{fulllineitems}



\section{\texttt{RESThandlers.Streetlight} Streetlight REST handler}
\label{handler/handlers:module-RESThandlers.Streetlight.Handler}\label{handler/handlers::doc}\label{handler/handlers:resthandlers-streetlight-streetlight-rest-handler}\index{RESThandlers.Streetlight.Handler (module)}\phantomsection\label{handler/handlers:module-Handlers.Streetlight.handler}\index{Handlers.Streetlight.handler (module)}\index{StreetlightHandler (class in RESThandlers.Streetlight.Handler)}

\begin{fulllineitems}
\phantomsection\label{handler/handlers:RESThandlers.Streetlight.Handler.StreetlightHandler}\pysigline{\strong{class }\code{RESThandlers.Streetlight.Handler.}\bfcode{StreetlightHandler}}
Bases: {\hyperref[handler/interface:RESThandlers.HandlerInterface.HandlerBaseClass.HandlerBase]{\code{RESThandlers.HandlerInterface.HandlerBaseClass.HandlerBase}}}
\index{delete\_all() (RESThandlers.Streetlight.Handler.StreetlightHandler method)}

\begin{fulllineitems}
\phantomsection\label{handler/handlers:RESThandlers.Streetlight.Handler.StreetlightHandler.delete_all}\pysiglinewithargsret{\bfcode{delete\_all}}{}{}
\end{fulllineitems}

\index{delete\_near() (RESThandlers.Streetlight.Handler.StreetlightHandler method)}

\begin{fulllineitems}
\phantomsection\label{handler/handlers:RESThandlers.Streetlight.Handler.StreetlightHandler.delete_near}\pysiglinewithargsret{\bfcode{delete\_near}}{\emph{latitude}, \emph{longitude}, \emph{nrange}}{}
\end{fulllineitems}

\index{get\_all() (RESThandlers.Streetlight.Handler.StreetlightHandler method)}

\begin{fulllineitems}
\phantomsection\label{handler/handlers:RESThandlers.Streetlight.Handler.StreetlightHandler.get_all}\pysiglinewithargsret{\bfcode{get\_all}}{\emph{mini=True}}{}
\end{fulllineitems}

\index{get\_by\_id() (RESThandlers.Streetlight.Handler.StreetlightHandler method)}

\begin{fulllineitems}
\phantomsection\label{handler/handlers:RESThandlers.Streetlight.Handler.StreetlightHandler.get_by_id}\pysiglinewithargsret{\bfcode{get\_by\_id}}{\emph{iid}}{}
\end{fulllineitems}

\index{get\_field\_names() (RESThandlers.Streetlight.Handler.StreetlightHandler method)}

\begin{fulllineitems}
\phantomsection\label{handler/handlers:RESThandlers.Streetlight.Handler.StreetlightHandler.get_field_names}\pysiglinewithargsret{\bfcode{get\_field\_names}}{}{}
\end{fulllineitems}

\index{get\_item\_count() (RESThandlers.Streetlight.Handler.StreetlightHandler method)}

\begin{fulllineitems}
\phantomsection\label{handler/handlers:RESThandlers.Streetlight.Handler.StreetlightHandler.get_item_count}\pysiglinewithargsret{\bfcode{get\_item\_count}}{}{}
\end{fulllineitems}

\index{get\_near() (RESThandlers.Streetlight.Handler.StreetlightHandler method)}

\begin{fulllineitems}
\phantomsection\label{handler/handlers:RESThandlers.Streetlight.Handler.StreetlightHandler.get_near}\pysiglinewithargsret{\bfcode{get\_near}}{\emph{longitude}, \emph{latitude}, \emph{nrange=0.001}, \emph{mini=False}}{}
\end{fulllineitems}

\index{get\_within\_rectangle() (RESThandlers.Streetlight.Handler.StreetlightHandler method)}

\begin{fulllineitems}
\phantomsection\label{handler/handlers:RESThandlers.Streetlight.Handler.StreetlightHandler.get_within_rectangle}\pysiglinewithargsret{\bfcode{get\_within\_rectangle}}{\emph{xtop\_right}, \emph{ytop\_right}, \emph{xbottom\_left}, \emph{ybottom\_left}, \emph{mini=False}}{}
\end{fulllineitems}

\index{handler\_id (RESThandlers.Streetlight.Handler.StreetlightHandler attribute)}

\begin{fulllineitems}
\phantomsection\label{handler/handlers:RESThandlers.Streetlight.Handler.StreetlightHandler.handler_id}\pysigline{\bfcode{handler\_id}}
staticmethod(function) -\textgreater{} method

Convert a function to be a static method.

A static method does not receive an implicit first argument.
To declare a static method, use this idiom:
\begin{quote}

class C:
def f(arg1, arg2, ...): ...
f = staticmethod(f)
\end{quote}

It can be called either on the class (e.g. C.f()) or on an instance
(e.g. C().f()).  The instance is ignored except for its class.

Static methods in Python are similar to those found in Java or C++.
For a more advanced concept, see the classmethod builtin.

\end{fulllineitems}

\index{search() (RESThandlers.Streetlight.Handler.StreetlightHandler method)}

\begin{fulllineitems}
\phantomsection\label{handler/handlers:RESThandlers.Streetlight.Handler.StreetlightHandler.search}\pysiglinewithargsret{\bfcode{search}}{\emph{regex}, \emph{limit}, \emph{field=None}}{}
\end{fulllineitems}

\index{update\_db() (RESThandlers.Streetlight.Handler.StreetlightHandler method)}

\begin{fulllineitems}
\phantomsection\label{handler/handlers:RESThandlers.Streetlight.Handler.StreetlightHandler.update_db}\pysiglinewithargsret{\bfcode{update\_db}}{}{}
\end{fulllineitems}


\end{fulllineitems}

\phantomsection\label{handler/handlers:module-RESThandlers.Streetlight.models}\index{RESThandlers.Streetlight.models (module)}\phantomsection\label{handler/handlers:module-Handlers.Streetlight.models}\index{Handlers.Streetlight.models (module)}

\section{\texttt{RESThandlers.Playgrounds} Playgrounds REST handler}
\label{handler/handlers:module-RESThandlers.Playgrounds.Handler}\label{handler/handlers:resthandlers-playgrounds-playgrounds-rest-handler}\index{RESThandlers.Playgrounds.Handler (module)}\phantomsection\label{handler/handlers:module-Handlers.Playgrounds.handler}\index{Handlers.Playgrounds.handler (module)}\index{PlaygroundHandler (class in RESThandlers.Playgrounds.Handler)}

\begin{fulllineitems}
\phantomsection\label{handler/handlers:RESThandlers.Playgrounds.Handler.PlaygroundHandler}\pysigline{\strong{class }\code{RESThandlers.Playgrounds.Handler.}\bfcode{PlaygroundHandler}}
Bases: {\hyperref[handler/interface:RESThandlers.HandlerInterface.HandlerBaseClass.HandlerBase]{\code{RESThandlers.HandlerInterface.HandlerBaseClass.HandlerBase}}}
\index{delete\_all() (RESThandlers.Playgrounds.Handler.PlaygroundHandler method)}

\begin{fulllineitems}
\phantomsection\label{handler/handlers:RESThandlers.Playgrounds.Handler.PlaygroundHandler.delete_all}\pysiglinewithargsret{\bfcode{delete\_all}}{}{}
\end{fulllineitems}

\index{delete\_near() (RESThandlers.Playgrounds.Handler.PlaygroundHandler method)}

\begin{fulllineitems}
\phantomsection\label{handler/handlers:RESThandlers.Playgrounds.Handler.PlaygroundHandler.delete_near}\pysiglinewithargsret{\bfcode{delete\_near}}{\emph{latitude}, \emph{longitude}, \emph{nrange}}{}
\end{fulllineitems}

\index{get\_all() (RESThandlers.Playgrounds.Handler.PlaygroundHandler method)}

\begin{fulllineitems}
\phantomsection\label{handler/handlers:RESThandlers.Playgrounds.Handler.PlaygroundHandler.get_all}\pysiglinewithargsret{\bfcode{get\_all}}{\emph{mini=True}}{}
\end{fulllineitems}

\index{get\_by\_id() (RESThandlers.Playgrounds.Handler.PlaygroundHandler method)}

\begin{fulllineitems}
\phantomsection\label{handler/handlers:RESThandlers.Playgrounds.Handler.PlaygroundHandler.get_by_id}\pysiglinewithargsret{\bfcode{get\_by\_id}}{\emph{iid}}{}
\end{fulllineitems}

\index{get\_field\_names() (RESThandlers.Playgrounds.Handler.PlaygroundHandler method)}

\begin{fulllineitems}
\phantomsection\label{handler/handlers:RESThandlers.Playgrounds.Handler.PlaygroundHandler.get_field_names}\pysiglinewithargsret{\bfcode{get\_field\_names}}{}{}
\end{fulllineitems}

\index{get\_item\_count() (RESThandlers.Playgrounds.Handler.PlaygroundHandler method)}

\begin{fulllineitems}
\phantomsection\label{handler/handlers:RESThandlers.Playgrounds.Handler.PlaygroundHandler.get_item_count}\pysiglinewithargsret{\bfcode{get\_item\_count}}{}{}
\end{fulllineitems}

\index{get\_near() (RESThandlers.Playgrounds.Handler.PlaygroundHandler method)}

\begin{fulllineitems}
\phantomsection\label{handler/handlers:RESThandlers.Playgrounds.Handler.PlaygroundHandler.get_near}\pysiglinewithargsret{\bfcode{get\_near}}{\emph{longitude}, \emph{latitude}, \emph{nrange=0.001}, \emph{mini=False}}{}
\end{fulllineitems}

\index{get\_within\_rectangle() (RESThandlers.Playgrounds.Handler.PlaygroundHandler method)}

\begin{fulllineitems}
\phantomsection\label{handler/handlers:RESThandlers.Playgrounds.Handler.PlaygroundHandler.get_within_rectangle}\pysiglinewithargsret{\bfcode{get\_within\_rectangle}}{\emph{xtop\_right}, \emph{ytop\_right}, \emph{xbottom\_left}, \emph{ybottom\_left}, \emph{mini=False}}{}
\end{fulllineitems}

\index{handler\_id (RESThandlers.Playgrounds.Handler.PlaygroundHandler attribute)}

\begin{fulllineitems}
\phantomsection\label{handler/handlers:RESThandlers.Playgrounds.Handler.PlaygroundHandler.handler_id}\pysigline{\bfcode{handler\_id}}
staticmethod(function) -\textgreater{} method

Convert a function to be a static method.

A static method does not receive an implicit first argument.
To declare a static method, use this idiom:
\begin{quote}

class C:
def f(arg1, arg2, ...): ...
f = staticmethod(f)
\end{quote}

It can be called either on the class (e.g. C.f()) or on an instance
(e.g. C().f()).  The instance is ignored except for its class.

Static methods in Python are similar to those found in Java or C++.
For a more advanced concept, see the classmethod builtin.

\end{fulllineitems}

\index{search() (RESThandlers.Playgrounds.Handler.PlaygroundHandler method)}

\begin{fulllineitems}
\phantomsection\label{handler/handlers:RESThandlers.Playgrounds.Handler.PlaygroundHandler.search}\pysiglinewithargsret{\bfcode{search}}{\emph{regex}, \emph{limit}, \emph{field=None}}{}
\end{fulllineitems}

\index{update\_db() (RESThandlers.Playgrounds.Handler.PlaygroundHandler method)}

\begin{fulllineitems}
\phantomsection\label{handler/handlers:RESThandlers.Playgrounds.Handler.PlaygroundHandler.update_db}\pysiglinewithargsret{\bfcode{update\_db}}{}{}
\end{fulllineitems}


\end{fulllineitems}

\phantomsection\label{handler/handlers:module-RESThandlers.Playgrounds.models}\index{RESThandlers.Playgrounds.models (module)}\phantomsection\label{handler/handlers:module-Handlers.Playgrounds.models}\index{Handlers.Playgrounds.models (module)}

\chapter{Indices and tables}
\label{index:indices-and-tables}\begin{itemize}
\item {} 
\emph{genindex}

\item {} 
\emph{modindex}

\item {} 
\emph{search}

\end{itemize}


\renewcommand{\indexname}{Python Module Index}
\begin{theindex}
\def\bigletter#1{{\Large\sffamily#1}\nopagebreak\vspace{1mm}}
\bigletter{h}
\item {\texttt{Handlers.Interface.base}} \emph{(Unix, Windows)}, \pageref{handler/interface:module-Handlers.Interface.base}
\item {\texttt{Handlers.Interface.Exceptions}} \emph{(Unix, Windows)}, \pageref{handler/interface:module-Handlers.Interface.Exceptions}
\item {\texttt{Handlers.Interface.Factory}} \emph{(Unix, Windows)}, \pageref{handler/factory:module-Handlers.Interface.Factory}
\item {\texttt{Handlers.Playgrounds.handler}} \emph{(Unix, Windows)}, \pageref{handler/handlers:module-Handlers.Playgrounds.handler}
\item {\texttt{Handlers.Playgrounds.models}} \emph{(Unix, Windows)}, \pageref{handler/handlers:module-Handlers.Playgrounds.models}
\item {\texttt{Handlers.Streetlight.handler}} \emph{(Unix, Windows)}, \pageref{handler/handlers:module-Handlers.Streetlight.handler}
\item {\texttt{Handlers.Streetlight.models}} \emph{(Unix, Windows)}, \pageref{handler/handlers:module-Handlers.Streetlight.models}
\indexspace
\bigletter{l}
\item {\texttt{lbd\_backend.LBD\_REST\_locationdata.decorators}}, \pageref{codedoc/decdoc:module-lbd_backend.LBD_REST_locationdata.decorators}
\item {\texttt{lbd\_backend.LBD\_REST\_locationdata.models}}, \pageref{codedoc/locdoc:module-lbd_backend.LBD_REST_locationdata.models}
\item {\texttt{lbd\_backend.LBD\_REST\_locationdata.views}}, \pageref{codedoc/locdoc:module-lbd_backend.LBD_REST_locationdata.views}
\item {\texttt{lbd\_backend.LBD\_REST\_messagedata.models}}, \pageref{codedoc/msgdoc:module-lbd_backend.LBD_REST_messagedata.models}
\item {\texttt{lbd\_backend.LBD\_REST\_messagedata.views}}, \pageref{codedoc/msgdoc:module-lbd_backend.LBD_REST_messagedata.views}
\item {\texttt{LocationdataREST.decorators}} \emph{(Unix, Windows)}, \pageref{codedoc/decdoc:module-LocationdataREST.decorators}
\item {\texttt{LocationdataREST.models}} \emph{(Unix, Windows)}, \pageref{codedoc/locdoc:module-LocationdataREST.models}
\item {\texttt{LocationdataREST.views}} \emph{(Unix, Windows)}, \pageref{codedoc/locdoc:module-LocationdataREST.views}
\indexspace
\bigletter{m}
\item {\texttt{MessagedataREST.models}} \emph{(Unix, Windows)}, \pageref{codedoc/msgdoc:module-MessagedataREST.models}
\item {\texttt{MessagedataREST.views}} \emph{(Unix, Windows)}, \pageref{codedoc/msgdoc:module-MessagedataREST.views}
\indexspace
\bigletter{r}
\item {\texttt{RESThandlers.HandlerInterface.Exceptions}}, \pageref{handler/interface:module-RESThandlers.HandlerInterface.Exceptions}
\item {\texttt{RESThandlers.HandlerInterface.Factory}}, \pageref{handler/factory:module-RESThandlers.HandlerInterface.Factory}
\item {\texttt{RESThandlers.HandlerInterface.HandlerBaseClass}}, \pageref{handler/interface:module-RESThandlers.HandlerInterface.HandlerBaseClass}
\item {\texttt{RESThandlers.Playgrounds.Handler}}, \pageref{handler/handlers:module-RESThandlers.Playgrounds.Handler}
\item {\texttt{RESThandlers.Playgrounds.models}}, \pageref{handler/handlers:module-RESThandlers.Playgrounds.models}
\item {\texttt{RESThandlers.Streetlight.Handler}}, \pageref{handler/handlers:module-RESThandlers.Streetlight.Handler}
\item {\texttt{RESThandlers.Streetlight.models}}, \pageref{handler/handlers:module-RESThandlers.Streetlight.models}
\end{theindex}

\renewcommand{\indexname}{Index}
\printindex
\end{document}
